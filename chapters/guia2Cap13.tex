\chapter{Entrega Final, Demonstração Pública e Checklist de Avaliação}

Este capítulo descreve de forma detalhada como preparar a entrega final do
projeto, organizar a apresentação pública, documentar o produto e verificar
todos os critérios necessários para obter uma avaliação completa e bem-sucedida.

A entrega final não é apenas mostrar o sistema funcionando: é provar que o
grupo planejou, construiu, organizou, documentou e validou cada parte do
trabalho.  
Equipes iniciantes aprendem muito nesse processo — especialmente sobre
responsabilidade técnica e comunicação.

%%%%%%%%%%%%%%%%%%%%%%%%%%%%%%%%%%%%%%%%%%%%%%%%%%%%%%%%%%%%%%%%%%%%
\section*{1. Preparação da Versão Final (Release)}

A versão final do software deve ser estável, revisada e empacotada para uso.
Recomenda-se:

\begin{itemize}
	\item congelar o código (code freeze) poucos dias antes da entrega;
	\item testar todas as funcionalidades do início ao fim;
	\item corrigir erros críticos imediatamente;
	\item gerar uma versão numerada (ex.: v1.0);
	\item salvar executável, banco de dados e arquivos extras.
\end{itemize}

A versão deve ser entregue de forma organizada:

\begin{itemize}
	\item pasta ``\gls{release}'';
	\item instruções de instalação;
	\item documentação do usuário;
	\item prints das telas;
	\item vídeo da aplicação funcionando (opcional, mas recomendado).
\end{itemize}

%%%%%%%%%%%%%%%%%%%%%%%%%%%%%%%%%%%%%%%%%%%%%%%%%%%%%%%%%%%%%%%%%%%%
\section*{2. Organização da Apresentação Pública}

A apresentação pública demonstra não só o sistema, mas também a capacidade da
equipe de explicar, justificar e defender suas escolhas.

A apresentação deve conter:

\begin{itemize}
	\item introdução ao problema;
	\item resumo das funcionalidades;
	\item arquitetura geral;
	\item pontos fortes do sistema;
	\item demonstração ao vivo;
	\item dificuldades enfrentadas;
	\item como a equipe lidou com problemas;
	\item conclusão e próximos passos.
\end{itemize}

%%%%%%%%%%%%%%%%%%%%%%%%%%%%%%%%%%%%%%%%%%%%%%%%%%%%%%%%%%%%%%%%%%%%
\section*{3. Demonstração ao Vivo (Live Demo)}

A demonstração prática é o ponto mais observado durante a avaliação.

Passos essenciais:

\begin{itemize}
	\item iniciar a demo em uma versão limpa do sistema;
	\item evitar dados antigos confusos;
	\item mostrar os fluxos principais:
	cadastro, consulta, edição e exclusão;
	\item explicar cada passo de forma simples;
	\item manter calma se algo der errado;
	\item ter um plano B (prints, vídeo ou passo a passo).
\end{itemize}

Erros comuns em demos:

\begin{itemize}
	\item esquecer login e senha;
	\item executar versão desatualizada;
	\item usar computadores diferentes sem testar antes;
	\item não saber explicar uma parte do código;
	\item não ter exemplos reais para mostrar.
\end{itemize}

%%%%%%%%%%%%%%%%%%%%%%%%%%%%%%%%%%%%%%%%%%%%%%%%%%%%%%%%%%%%%%%%%%%%
\section*{4. Documentação Final do Projeto}

A documentação é parte essencial da nota e deve incluir:

\begin{itemize}
	\item descrição geral do sistema;
	\item requisitos funcionais e não funcionais;
	\item arquitetura e fluxos;
	\item modelo de dados;
	\item manual do usuário;
	\item manual de instalação;
	\item prints organizados;
	\item relatório final de desenvolvimento;
	\item lista de autores e funções de cada aluno.
\end{itemize}

%%%%%%%%%%%%%%%%%%%%%%%%%%%%%%%%%%%%%%%%%%%%%%%%%%%%%%%%%%%%%%%%%%%%
\section*{5. Portfólio dos Alunos}

A entrega final também serve como material de portfólio.

Cada aluno deve:

\begin{itemize}
	\item manter link para o repositório;
	\item destacar \glspl{issue} resolvidas;
	\item mostrar commits importantes;
	\item registrar papéis desempenhados;
	\item criar uma página simples descrevendo sua parte;
	\item gerar prints e vídeos de contribuições.
\end{itemize}

%%%%%%%%%%%%%%%%%%%%%%%%%%%%%%%%%%%%%%%%%%%%%%%%%%%%%%%%%%%%%%%%%%%%
\section*{6. Checklist Completo de Avaliação}

A seguir, um checklist que pode ser aplicado pelo professor e usado
pelos alunos para revisão final:

\subsection*{6.1 Funcionalidades}
\begin{itemize}
	\item todas as telas funcionam?
	\item os fluxos principais completam sem erros?
	\item ações críticas pedem confirmação?
	\item existe validação de entrada adequada?
\end{itemize}

\subsection*{6.2 Qualidade Técnica}
\begin{itemize}
	\item código organizado e legível?
	\item comentários essenciais presentes?
	\item padronização de nome de arquivos e funções?
	\item banco de dados normalizado?
	\item arquivos desnecessários removidos?
\end{itemize}

\subsection*{6.3 Robustez}
\begin{itemize}
	\item erros tratados corretamente?
	\item sistema não quebra com dados errados?
	\item autenticação e autorização funcionando?
	\item logs adequados?
\end{itemize}

\subsection*{6.4 Testes}
\begin{itemize}
	\item todos os testes previstos foram executados?
	\item \glspl{bug} registrados e corrigidos?
	\item testes de regressão feitos?
	\item prints anexados?
\end{itemize}

\subsection*{6.5 Documentação}
\begin{itemize}
	\item manual do usuário completo?
	\item manual de instalação testado?
	\item arquitetura explicada claramente?
	\item prints ilustrativos?
	\item relatório final organizado?
\end{itemize}

\subsection*{6.6 Apresentação}
\begin{itemize}
	\item demonstração funcionando?
	\item grupo calmo e organizado?
	\item divisão clara de falas?
	\item domínio do assunto?
	\item exemplos reais?
\end{itemize}

%%%%%%%%%%%%%%%%%%%%%%%%%%%%%%%%%%%%%%%%%%%%%%%%%%%%%%%%%%%%%%%%%%%%
\section*{7. Preparação Emocional e Organizacional dos Alunos (Base Científica)}

Projetos complexos em ambientes educacionais têm melhor desempenho quando os
alunos:

\begin{itemize}
	\item recebem instruções claras e previsíveis;
	\item têm oportunidade de treinar a demo várias vezes;
	\item reduzem a ansiedade conhecendo o roteiro da apresentação;
	\item entendem que falhas pequenas são normais;
	\item têm papéis definidos (locutor principal, operador, apoio).
\end{itemize}

Estratégias práticas:

\begin{itemize}
	\item simular a apresentação com outra dupla;
	\item pedir feedback específico;
	\item treinar antes de montar os slides;
	\item dividir o tempo igualmente entre membros do grupo.
\end{itemize}

%%%%%%%%%%%%%%%%%%%%%%%%%%%%%%%%%%%%%%%%%%%%%%%%%%%%%%%%%%%%%%%%%%%%
\section*{8. Instruções Passo-a-Passo (Para Leigos)}

\begin{enumerate}
	\item finalize o sistema e teste tudo.
	\item tire prints das telas funcionando.
	\item organize as pastas: \textit{\gls{release}}, \textit{documentação}, \textit{prints}.
	\item crie o manual de instalação:
	“clique aqui, depois aqui...”.
	\item treine a apresentação ao menos duas vezes.
	\item defina quem vai falar cada parte.
	\item prepare um plano B (vídeo ou prints).
	\item chegue cedo no dia da apresentação.
	\item abra o sistema e deixe tudo pronto na tela inicial.
	\item apresente com calma e clareza.
\end{enumerate}

%%%%%%%%%%%%%%%%%%%%%%%%%%%%%%%%%%%%%%%%%%%%%%%%%%%%%%%%%%%%%%%%%%%%
\section*{9. Encerramento do Capítulo}

A entrega final é a culminação de todo esforço do semestre.  
É quando o software ganha forma pública e se transforma em aprendizado real.
Ao dominar esse processo, os alunos desenvolvem não apenas habilidades
técnicas, mas também comunicação, organização e responsabilidade — competências
essenciais em qualquer carreira tecnológica.
