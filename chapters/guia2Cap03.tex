\chapter{Planejamento e Cronograma}
Este capítulo apresenta como organizar o trabalho de uma equipe iniciante,
estabelecendo um plano realista, dividido em etapas claras, com papéis definidos
e um cronograma compatível com o tempo disponível. Organizar bem esta fase reduz
retrabalho, dá direção ao time e evita que o projeto trave.

%%%%%%%%%%%%%%%%%%%%%%%%%%%%%%%%%%%%%%%%%%%%%%%%%%%%%%%%%%%%%%%%%%%%
\section*{1. Importância do Planejamento Inicial}

Planejar significa distribuir as tarefas no tempo, definir responsabilidades,
estimar esforço e antecipar dificuldades. Para iniciantes, o planejamento
garante:

\begin{itemize}
	\item ritmo constante de evolução;
	\item divisão clara de funções;
	\item foco no essencial;
	\item redução de ansiedade e confusão;
	\item maior previsibilidade e segurança.
\end{itemize}

Sem planejamento, grupos iniciantes tendem a:

\begin{itemize}
	\item começar pelo código antes de saber o que construir;
	\item ignorar dependências entre tarefas;
	\item deixar tudo para a última semana;
	\item perder tempo refazendo partes do projeto.
\end{itemize}

%%%%%%%%%%%%%%%%%%%%%%%%%%%%%%%%%%%%%%%%%%%%%%%%%%%%%%%%%%%%%%%%%%%%
\section*{2. Divisão do Projeto em Etapas Claras}

Todo software, independente do tema, pode ser dividido nas seguintes etapas:

\begin{enumerate}
	\item Definição do Problema e Requisitos.
	\item Modelagem e Arquitetura.
	\item Protótipo Inicial.
	\item Desenvolvimento do \ac{mvp}.
	\item Expansão e Melhorias.
	\item Testes e Correções.
	\item Documentação.
	\item Apresentação Final.
\end{enumerate}

Para alunos iniciantes, cada etapa deve ser curta, objetiva e com entregas
pequenas, evitando sobrecarga cognitiva.

%%%%%%%%%%%%%%%%%%%%%%%%%%%%%%%%%%%%%%%%%%%%%%%%%%%%%%%%%%%%%%%%%%%%
\section*{3. Estrutura de Grupos e Papéis da Equipe}

Mesmo com 20 alunos iniciante, é possível organizar um grupo eficiente dividindo
a turma em subequipes. Papéis sugeridos:

\subsection*{3.1 Equipe de Levantamento e Requisitos}
Anota problemas, escreve requisitos, revisa e valida documentos.

\subsection*{3.2 Equipe de Modelagem}
Cria modelos de dados, fluxos de tela e diagramas simples.

\subsection*{3.3 Equipe de \gls{frontend}}
Responsável pela interface visual.

\subsection*{3.4 Equipe de \gls{backend}}
Constrói rotas, lógica e integrações.

\subsection*{3.5 Equipe de Testes}
Cria roteiros de teste e executa validações.

\subsection*{3.6 Equipe de Documentação}
Escreve relatórios, manuais e prepara a apresentação final.

O professor pode reorganizar os alunos semanalmente, garantindo que todos passem
por todas as funções.

%%%%%%%%%%%%%%%%%%%%%%%%%%%%%%%%%%%%%%%%%%%%%%%%%%%%%%%%%%%%%%%%%%%%
\section*{4. Planejamento Baseado em Marcos (Milestones)}

Para organizar as etapas, definimos marcos de verificação:

\begin{itemize}
	\item \textbf{Marco 1}: Requisitos aprovados.
	\item \textbf{Marco 2}: Protótipo validado.
	\item \textbf{Marco 3}: \ac{mvp} funcionando.
	\item \textbf{Marco 4}: Versão quase final com testes.
	\item \textbf{Marco 5}: Entrega final documentada.
\end{itemize}

Esses marcos ajudam o professor e os alunos a enxergarem progresso real.

%%%%%%%%%%%%%%%%%%%%%%%%%%%%%%%%%%%%%%%%%%%%%%%%%%%%%%%%%%%%%%%%%%%%
\section*{5. Planejamento Sugerido para 4 Meses}

Com 4 horas de aula por semana durante 4 meses, temos aproximadamente:

\[
4 \text{ meses} \times 4 \text{ semanas} \times 4 \text{ horas} = 
\textbf{64 horas totais}
\]

Distribuição recomendada:

\begin{itemize}
	\item Semanas 1–3: Problema + Requisitos + Modelagem.
	\item Semanas 4–6: Protótipo + Arquitetura.
	\item Semanas 7–10: \ac{mvp}.
	\item Semanas 11–13: Expansões.
	\item Semanas 14–15: Testes.
	\item Semanas 16: Documentação + Ensaio da Apresentação.
\end{itemize}

Essa divisão dá tempo suficiente para aprender, praticar e revisar.

%%%%%%%%%%%%%%%%%%%%%%%%%%%%%%%%%%%%%%%%%%%%%%%%%%%%%%%%%%%%%%%%%%%%
\section*{6. Como Montar o Cronograma em Sala}

Recomenda-se que o cronograma siga estas diretrizes:

\begin{itemize}
	\item Cada tarefa deve durar de 30 minutos a 1 semana.
	\item Não criar tarefas grandes demais.
	\item Revisar o cronograma semanalmente.
	\item Atribuir responsáveis por tarefa.
	\item Mudar prazos quando necessário, mas nunca pular entregáveis.
\end{itemize}

Para iniciantes, a simplicidade do cronograma é crucial.

%%%%%%%%%%%%%%%%%%%%%%%%%%%%%%%%%%%%%%%%%%%%%%%%%%%%%%%%%%%%%%%%%%%%
\section*{7. Ferramentas Simples para Planejamento}

Ferramentas adequadas ao nível dos alunos:

\begin{itemize}
	\item Quadro \gls{kanban} no GitHub Projects.
	\item Planilha simples no Google Sheets.
	\item Lista de tarefas em papel ou quadro branco.
\end{itemize}

O objetivo não é usar ferramentas complexas, mas manter disciplina e organização.

%%%%%%%%%%%%%%%%%%%%%%%%%%%%%%%%%%%%%%%%%%%%%%%%%%%%%%%%%%%%%%%%%%%%
\section*{8. Indicadores Simples de Progresso}

Métricas fáceis para iniciantes:

\begin{itemize}
	\item número de tarefas concluídas por semana;
	\item tempo médio por tarefa;
	\item número de dúvidas levantadas por sprint;
	\item número de requisitos finalizados;
	\item quantidade de commits no repositório.
\end{itemize}

Essas métricas dão um senso claro de avanço.

%%%%%%%%%%%%%%%%%%%%%%%%%%%%%%%%%%%%%%%%%%%%%%%%%%%%%%%%%%%%%%%%%%%%
\section*{9. Erros Comuns no Planejamento}

\begin{itemize}
	\item Estimar tarefas com otimismo excessivo.
	\item Criar tarefas vagas e genéricas.
	\item Não revisar o plano semanalmente.
	\item Assumir que tudo será feito exatamente como planejado.
	\item Tentar fazer tudo de uma vez, sem dividir em partes.
\end{itemize}

%%%%%%%%%%%%%%%%%%%%%%%%%%%%%%%%%%%%%%%%%%%%%%%%%%%%%%%%%%%%%%%%%%%%
\section*{10. Checklist do Capítulo}

Ao completar este capítulo, o aluno deve conseguir:

\begin{itemize}
	\item montar um cronograma simples e realista;
	\item dividir o trabalho em etapas;
	\item criar milestones;
	\item distribuir papéis;
	\item acompanhar o andamento do projeto;
	\item prever dificuldades antes que aconteçam.
\end{itemize}

%%%%%%%%%%%%%%%%%%%%%%%%%%%%%%%%%%%%%%%%%%%%%%%%%%%%%%%%%%%%%%%%%%%%
\section*{11. Instruções Passo-a-Passo (Para Leigos)}

\begin{enumerate}
	\item Abra um caderno ou documento e divida-o em semanas.
	\item Liste os grandes blocos do projeto (requisitos, protótipo, \ac{mvp}…).
	\item Distribua esses blocos nos espaços semanais.
	\item Crie tarefas pequenas para cada semana.
	\item Defina quem ficará responsável por cada tarefa.
	\item Marque um dia fixo da semana para revisar tudo.
	\item Registre atrasos e antecipe soluções simples.
	\item Atualize o cronograma semanalmente.
\end{enumerate}

%%%%%%%%%%%%%%%%%%%%%%%%%%%%%%%%%%%%%%%%%%%%%%%%%%%%%%%%%%%%%%%%%%%%
\section*{12. Encerramento do Capítulo}

Um bom planejamento transforma um time iniciante em um time organizado.  
Ele serve como guia visual, reduz ansiedade e mantém todos orientados,
independentemente da velocidade individual.  
Com o cronograma bem estabelecido, o projeto ganha direção clara e ritmo estável.
