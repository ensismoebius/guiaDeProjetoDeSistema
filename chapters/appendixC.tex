\chapter{Apêndice C — Roteiro de Aula, Plano de Sprints e Checklist do Semestre}

Este apêndice fornece materiais diretamente aplicáveis em sala de aula, permitindo
que o professor gerencie de forma clara as 16 semanas (4 meses) de trabalho.

%%%%%%%%%%%%%%%%%%%%%%%%%%%%%%%%%%%%%%%%%%%%%%%%%%%%%%%%%%%%%%%%%%%%
\section*{1. Roteiro de Aula Completo (16 Semanas)}

\begin{enumerate}
	\item \textbf{Semana 1:} Apresentação do projeto, formação dos grupos,
	introdução a \ac{git} e estrutura do repositório.
	\item \textbf{Semana 2:} Levantamento de requisitos e primeira modelagem.
	\item \textbf{Semana 3:} Protótipo inicial de telas e arquitetura preliminar.
	\item \textbf{Semana 4:} Definição da \ac{api}, modelo de dados e fluxo geral.
	\item \textbf{Semana 5:} Início da implementação do \gls{backend}.
	\item \textbf{Semana 6:} Início do \gls{frontend} e integração inicial.
	\item \textbf{Semana 7:} Testes manuais e primeira revisão de código.
	\item \textbf{Semana 8:} Construção do \ac{mvp}.
	\item \textbf{Semana 9:} Segunda rodada de testes e refatoração.
	\item \textbf{Semana 10:} Implementação de funcionalidades complementares.
	\item \textbf{Semana 11:} Verificação de segurança e validações.
	\item \textbf{Semana 12:} Testes em fluxo completo (end-to-end).
	\item \textbf{Semana 13:} Documentação técnica e manual do usuário.
	\item \textbf{Semana 14:} Preparação da demo e ensaio.
	\item \textbf{Semana 15:} Entrega final e apresentação pública.
	\item \textbf{Semana 16:} Avaliação pós-projeto e reflexão da equipe.
\end{enumerate}

%%%%%%%%%%%%%%%%%%%%%%%%%%%%%%%%%%%%%%%%%%%%%%%%%%%%%%%%%%%%%%%%%%%%
\section*{2. Plano de Sprints (modelo pronto)}

Cada sprint dura 1 semana.

\subsection*{Sprint 1 — Organização}
	\begin{itemize}
			\item criar repositório;
			\item definir papéis do time;
			\item preparar ambiente;
			\item registrar primeiras \glspl{issue}.
	\end{itemize}

\subsection*{Sprint 2 — Requisitos}
	\begin{itemize}
			\item \glspl{userstory};
			\item restrições;
			\item protótipo simples das telas.
	\end{itemize}

\subsection*{Sprint 3 — Arquitetura}
	\begin{itemize}
			\item fluxos principais;
			\item modelo de dados;
			\item esquema da \ac{api}.
	\end{itemize}

\subsection*{Sprint 4 — \ac{mvp} \gls{backend}}
	\begin{itemize}
			\item rota principal;
			\item validações mínimas;
			\item testes iniciais.
	\end{itemize}

\subsection*{Sprint 5 — \ac{mvp} \gls{frontend}}
	\begin{itemize}
			\item telas funcionais;
			\item integração inicial;
			\item captura de erros.
	\end{itemize}

\subsection*{Sprint 6 — Refinamento}
	\begin{itemize}
			\item refatoração;
			\item revisão de código;
			\item correção de \glspl{bug}.
	\end{itemize}

\subsection*{Sprint 7 — Qualidade}
	\begin{itemize}
			\item testes completos;
			\item revisão da \ac{api};
			\item estabilização.
	\end{itemize}

\subsection*{Sprint 8 — Entrega Final}
	\begin{itemize}
			\item documentação;
			\item preparação da apresentação;
			\item \gls{release} final.
			\item (Opcional) Material para o professor.
	\end{itemize}

%%%%%%%%%%%%%%%%%%%%%%%%%%%%%%%%%%%%%%%%%%%%%%%%%%%%%%%%%%%%%%%%%%%%
\section*{3. Checklist Geral do Semestre}

\subsection*{Checklist Técnico}
	\begin{itemize}
			\item requisitos claros e documentados;
			\item arquitetura desenhada;
			\item \ac{api} definida e testada;
			\item banco de dados funcional;
			\item interface completa;
			\item validações implementadas;
			\item testes documentados;
			\item \glspl{bug} críticos resolvidos;
			\item manual de usuário pronto;
			\item \gls{release} final disponível.
	\end{itemize}

\subsection*{Checklist de Equipe}
	\begin{itemize}
			\item papéis definidos;
			\item reuniões semanais realizadas;
			\item \glspl{issue} abertas e fechadas regularmente;
			\item comunicação clara no repositório;
			\item revisão de código aplicada;
			\item conflitos resolvidos rapidamente.
	\end{itemize}

\subsection*{Checklist de Apresentação}
	\begin{itemize}
			\item roteiro claro;
			\item tempo ensaiado;
			\item demo pronta e testada;
			\item prints como plano B;
			\item domínio do sistema;
			\item divisão equilibrada das falas.
	\end{itemize}

%%%%%%%%%%%%%%%%%%%%%%%%%%%%%%%%%%%%%%%%%%%%%%%%%%%%%%%%%%%%%%%%%%%%
\section*{4. Material Extra para o Professor}

Sugestões práticas:

	\begin{itemize}
			\item usar rubricas de avaliação semanais;
			\item criar painéis visuais de progresso;
			\item promover sessões rápidas de mentoria por grupo;
			\item incentivar alunos a registrar microvitórias;
			\item intercalar teoria e prática a cada aula.
	\end{itemize}

%%%%%%%%%%%%%%%%%%%%%%%%%%%%%%%%%%%%%%%%%%%%%%%%%%%%%%%%%%%%%%%%%%%%
\section*{5. Encerramento do Apêndice}

Este apêndice fornece um conjunto completo de ferramentas operacionais para
organizar, conduzir e avaliar o projeto ao longo de quatro meses, garantindo
clareza, estrutura, ritmo e coerência pedagógica.

\section*{Template: Pull Request}

\begin{verbatim}
	# O que foi feito
	# Como testar
	# Checklist
\end{verbatim}
