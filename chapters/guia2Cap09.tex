\chapter{Gestão de Riscos e Continuidade}
Nenhum projeto de software está livre de problemas.  
Erros acontecem, prazos apertam, requisitos mudam, alunos faltam, ferramentas
travam e imprevistos aparecem sem aviso.  
Por isso, a gestão de riscos é essencial: ela permite antecipar falhas,
reduzir impacto e manter o projeto funcionando mesmo quando algo dá errado.

Para alunos iniciantes, aprender a lidar com riscos desenvolve maturidade,
organização e resiliência emocional no desenvolvimento.

%%%%%%%%%%%%%%%%%%%%%%%%%%%%%%%%%%%%%%%%%%%%%%%%%%%%%%%%%%%%%%%%%%%%
\section*{1. O Que é um Risco em Projetos de Software?}

Risco é qualquer coisa que possa prejudicar o andamento do projeto, causando:

\begin{itemize}
	\item atrasos;
	\item retrabalho;
	\item perda de qualidade;
	\item falhas no sistema;
	\item falta de motivação na equipe.
\end{itemize}

Gerenciar riscos significa:

\begin{enumerate}
	\item identificar o risco;
	\item analisar sua gravidade;
	\item agir antes que ele aconteça;
	\item criar um plano B.
\end{enumerate}

%%%%%%%%%%%%%%%%%%%%%%%%%%%%%%%%%%%%%%%%%%%%%%%%%%%%%%%%%%%%%%%%%%%%
\section*{2. Riscos Comuns em Turmas de Alunos Iniciantes}

Os riscos mais frequentes incluem:

\begin{itemize}
	\item alunos ausentes em dias importantes;
	\item tarefas grandes demais para o nível da turma;
	\item falta de comunicação entre grupos;
	\item atrasos acumulados por não testar;
	\item perda de arquivos ou commits;
	\item dificuldades técnicas não previstas;
	\item procrastinação coletiva;
	\item indecisão sobre funcionalidades.
\end{itemize}

Reconhecer esses riscos antecipadamente aumenta a chance de sucesso.

%%%%%%%%%%%%%%%%%%%%%%%%%%%%%%%%%%%%%%%%%%%%%%%%%%%%%%%%%%%%%%%%%%%%
\section*{3. Registro de Riscos (Risk Log)}

Um Risk Log é uma lista organizada com os principais riscos.

Cada item deve registrar:

\begin{itemize}
	\item descrição do risco;
	\item probabilidade (baixa/média/alta);
	\item impacto (baixo/médio/alto);
	\item responsável;
	\item plano de mitigação;
	\item plano de contingência.
\end{itemize}

Exemplo simples:

\begin{verbatim}
	Risco: Aluno responsável por módulo faltar.
	Probabilidade: alta
	Impacto: alto
	Mitigação: sempre trabalhar em dupla
	Contingência: reatribuir tarefa ao grupo ao lado
\end{verbatim}

%%%%%%%%%%%%%%%%%%%%%%%%%%%%%%%%%%%%%%%%%%%%%%%%%%%%%%%%%%%%%%%%%%%%
\section*{4. Como Avaliar Gravidade de um Risco}

Para iniciantes, a avaliação pode ser feita respondendo:

\begin{itemize}
	\item É provável que isso aconteça?
	\item Se acontecer, vai atrapalhar muito?
\end{itemize}

Exemplo:

\begin{itemize}
	\item alta probabilidade + alto impacto = risco crítico;
	\item baixa probabilidade + baixo impacto = risco mínimo.
\end{itemize}

%%%%%%%%%%%%%%%%%%%%%%%%%%%%%%%%%%%%%%%%%%%%%%%%%%%%%%%%%%%%%%%%%%%%
\section*{5. Estratégias de Mitigação}

Mitigar significa reduzir a chance do risco acontecer.

Exemplos:

\begin{itemize}
	\item trabalhar sempre em duplas;
	\item revisar tarefas semanalmente;
	\item quebrar tarefas grandes em partes pequenas;
	\item salvar commits frequentes;
	\item documentar tudo que foi decidido;
	\item repetir testes a cada pequena mudança.
\end{itemize}

%%%%%%%%%%%%%%%%%%%%%%%%%%%%%%%%%%%%%%%%%%%%%%%%%%%%%%%%%%%%%%%%%%%%
\section*{6. Estratégias de Contingência}

Contingência é o plano B: o que fazer se o risco já aconteceu.

Exemplos:

\begin{itemize}
	\item se alguém faltar, reatribuir tarefa ao colega da dupla;
	\item se um arquivo for perdido, restaurar \gls{commit} anterior;
	\item se uma funcionalidade falhar, usar versão simplificada;
	\item se o prazo estourar, remover funcionalidades não essenciais.
\end{itemize}

%%%%%%%%%%%%%%%%%%%%%%%%%%%%%%%%%%%%%%%%%%%%%%%%%%%%%%%%%%%%%%%%%%%%
\section*{7. Continuidade do Projeto (Business Continuity)}

Mesmo em projetos escolares, vale a pena planejar a continuidade:

\begin{itemize}
	\item manter o repositório sempre atualizado;
	\item ter mais de um aluno capaz de explicar cada módulo;
	\item deixar instruções claras no README;
	\item garantir que outro aluno consiga assumir a tarefa rapidamente;
	\item registrar tudo em documentos acessíveis.
\end{itemize}

Continuidade significa que o projeto não depende de uma única pessoa.

%%%%%%%%%%%%%%%%%%%%%%%%%%%%%%%%%%%%%%%%%%%%%%%%%%%%%%%%%%%%%%%%%%%%
\section*{8. Reuniões Semanais de Risco}

Sugestão de rotina:

\begin{enumerate}
	\item revisar o que foi feito;
	\item verificar o que deu errado;
	\item identificar novos riscos;
	\item ajustar prioridades;
	\item planejar ações corretivas.
\end{enumerate}

Essas reuniões devem ser rápidas (10 minutos).

%%%%%%%%%%%%%%%%%%%%%%%%%%%%%%%%%%%%%%%%%%%%%%%%%%%%%%%%%%%%%%%%%%%%
\section*{9. Indicadores Simples para Monitoramento}

Para iniciantes, use indicadores fáceis:

\begin{itemize}
	\item tarefas atrasadas por semana;
	\item número de \glspl{bug} encontrados;
	\item commits por semana;
	\item tempo médio para corrigir problemas;
	\item módulos sem responsáveis.
\end{itemize}

%%%%%%%%%%%%%%%%%%%%%%%%%%%%%%%%%%%%%%%%%%%%%%%%%%%%%%%%%%%%%%%%%%%%
\section*{10. Erros Comuns na Gestão de Riscos}

\begin{itemize}
	\item ignorar sinais de alerta;
	\item deixar riscos críticos sem responsável;
	\item esconder problemas por medo de avaliação;
	\item deixar tudo para o fim do semestre;
	\item assumir que “vai dar certo” sem plano.
\end{itemize}

%%%%%%%%%%%%%%%%%%%%%%%%%%%%%%%%%%%%%%%%%%%%%%%%%%%%%%%%%%%%%%%%%%%%
\section*{11. Checklist do Capítulo}

\begin{itemize}
	\item os riscos foram identificados?
	\item existe um Risk Log atualizado?
	\item o time sabe quem é responsável por cada risco?
	\item existem estratégias claras de mitigação?
	\item o grupo sabe o que fazer se o risco acontecer?
	\item há reuniões semanais para revisão?
\end{itemize}

%%%%%%%%%%%%%%%%%%%%%%%%%%%%%%%%%%%%%%%%%%%%%%%%%%%%%%%%%%%%%%%%%%%%
\section*{12. Instruções Passo-a-Passo (Para Leigos)}

\begin{enumerate}
	\item Pegue uma folha ou documento.
	\item Liste tudo que pode dar errado.
	\item Marque os riscos mais prováveis.
	\item Marque os riscos mais perigosos.
	\item Crie para cada risco: um plano A e um plano B.
	\item Escolha responsáveis para acompanhar cada item.
	\item Revise a lista toda semana.
\end{enumerate}

%%%%%%%%%%%%%%%%%%%%%%%%%%%%%%%%%%%%%%%%%%%%%%%%%%%%%%%%%%%%%%%%%%%%
\section*{13. Encerramento do Capítulo}

Gerenciar riscos ensina os alunos a lidar com imprevistos, planejar melhor,
antecipar problemas e manter a estabilidade do projeto.  
Em qualquer projeto profissional, essa habilidade é valorizada — e quanto mais cedo for praticada, melhor.
