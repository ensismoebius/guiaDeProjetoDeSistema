\chapter{Comunicação, Colaboração e Governança do Projeto}

Em projetos de software, especialmente quando desenvolvidos por turmas com muitos alunos, a qualidade da comunicação é mais determinante que a qualidade do código.  
Falhas de comunicação geram retrabalho, atrasos, conflitos, perda de motivação e funcionalidades implementadas de forma incorreta.  
Por outro lado, comunicação clara, colaboração organizada e governança estruturada transformam grupos iniciantes em equipes capazes de entregar software real.

Este capítulo ensina como organizar o trabalho em equipe, como comunicar-se corretamente e como manter o projeto funcionando de forma ordenada ao longo do semestre.

%%%%%%%%%%%%%%%%%%%%%%%%%%%%%%%%%%%%%%%%%%%%%%%%%%%%%%%%%%%%%%%%%%%%
\section*{1. Fundamentos da Comunicação em Equipes de Software}

Comunicação eficiente envolve:

\begin{itemize}
	\item clareza na explicação;
	\item registro escrito das decisões;
	\item repetição e confirmação do entendimento;
	\item uso disciplinado de ferramentas (GitHub, listas de tarefas, etc.);
	\item evitar suposições;
	\item pedir ajuda ao invés de esconder dúvidas.
\end{itemize}

Para iniciantes, comunicação clara reduz ansiedade e cria ambiente colaborativo.

%%%%%%%%%%%%%%%%%%%%%%%%%%%%%%%%%%%%%%%%%%%%%%%%%%%%%%%%%%%%%%%%%%%%
\section*{2. Canais de Comunicação Recomendados}

Cada tipo de mensagem deve usar o canal adequado:

\begin{itemize}
	\item \textbf{GitHub Issues}: tarefas, bugs, solicitações de mudança.
	\item \textbf{GitHub Commits}: alterações realizadas no código.
	\item \textbf{Mensagens instantâneas}: dúvidas rápidas e avisos urgentes.
	\item \textbf{Reuniões presenciais ou on-line}: alinhamentos semanais.
	\item \textbf{Documentos compartilhados}: decisões importantes e atas.
\end{itemize}

Regras gerais:

\begin{itemize}
	\item nunca deixar decisões importantes só em conversa informal;
	\item tudo que afeta código deve ser registrado em issue ou commit;
	\item mensagens devem ser objetivas, educadas e informativas.
\end{itemize}

%%%%%%%%%%%%%%%%%%%%%%%%%%%%%%%%%%%%%%%%%%%%%%%%%%%%%%%%%%%%%%%%%%%%
\section*{3. Como Escrever Mensagens Claras}

Para alunos iniciantes, recomenda-se:

\begin{itemize}
	\item frase curta e direta;
	\item evitar termos vagos (“quase pronto”, “tá dando erro”);
	\item descrever o que tentou fazer;
	\item adicionar print da tela ou trecho de código relevante;
	\item listar passos para reproduzir o problema.
\end{itemize}

Exemplo ruim:

\begin{verbatim}
	Não funciona aqui.
\end{verbatim}

Exemplo bom:

\begin{verbatim}
	Erro ao cadastrar cliente.
	Passos:
	1. Abrir tela de cadastro.
	2. Preencher nome e CPF.
	3. Clicar "Salvar".
	
	Resultado esperado: salvar os dados.
	Resultado obtido: tela trava e nada é salvo.
	
	Commit relacionado: a1b2c3.
\end{verbatim}

%%%%%%%%%%%%%%%%%%%%%%%%%%%%%%%%%%%%%%%%%%%%%%%%%%%%%%%%%%%%%%%%%%%%
\section*{4. Como Dar e Receber Feedback}

Feedback profissional:

\begin{itemize}
	\item foca no comportamento, não na pessoa;
	\item usa exemplos concretos;
	\item é direto e objetivo;
	\item oferece caminho de melhoria.
\end{itemize}

Para iniciantes, o professor deve reforçar que:

\begin{itemize}
	\item pedir ajuda é sinal de maturidade;
	\item feedback não é crítica pessoal;
	\item corrigir rapidamente é mais importante que acertar de primeira.
\end{itemize}

%%%%%%%%%%%%%%%%%%%%%%%%%%%%%%%%%%%%%%%%%%%%%%%%%%%%%%%%%%%%%%%%%%%%
\section*{5. Colaboração usando Git e GitHub}

Principais práticas:

\begin{itemize}
	\item criar uma branch por funcionalidade;
	\item abrir pull requests (PRs) explicando a mudança;
	\item revisar código do colega antes de aprovar;
	\item escrever commits com mensagens descritivas;
	\item sincronizar o código frequentemente.
\end{itemize}

Para iniciantes:

\begin{itemize}
	\item usar Pull Request como ferramenta de aprendizado;
	\item revisar código em dupla;
	\item evitar commits muito grandes.
\end{itemize}

%%%%%%%%%%%%%%%%%%%%%%%%%%%%%%%%%%%%%%%%%%%%%%%%%%%%%%%%%%%%%%%%%%%%
\section*{6. Divisão de Papéis no Projeto}

Sugestão de papéis:

\begin{itemize}
	\item \textbf{Líder Técnico}: organiza o desenvolvimento e integra código.
	\item \textbf{Gestor de Tarefas}: atualiza quadro Kanban e issues.
	\item \textbf{Responsáveis por Módulo}: cuidam de funcionalidades específicas.
	\item \textbf{Documentadores}: atualizam manuais e relatórios.
	\item \textbf{Testadores}: executam casos de teste.
\end{itemize}

Os papéis devem:

\begin{itemize}
	\item ser rotativos;
	\item evitar concentração de responsabilidade em poucas pessoas;
	\item ensinar todos a trabalhar em diferentes funções.
\end{itemize}

%%%%%%%%%%%%%%%%%%%%%%%%%%%%%%%%%%%%%%%%%%%%%%%%%%%%%%%%%%%%%%%%%%%%
\section*{7. Governança: Regras para Manter a Ordem}

Governança é o conjunto de regras que mantém o projeto funcionando bem.

Recomenda-se:

\begin{itemize}
	\item todas as tarefas devem ter responsável e prazo;
	\item nada é alterado na \textit{branch main} sem PR aprovado;
	\item reuniões semanais obrigatórias;
	\item código deve ser comentado;
	\item decisões técnicas devem ser documentadas em arquivo próprio;
	\item conflitos são resolvidos por discussão técnica com dados;
	\item cada módulo deve ter pelo menos duas pessoas que o conheçam.
\end{itemize}

%%%%%%%%%%%%%%%%%%%%%%%%%%%%%%%%%%%%%%%%%%%%%%%%%%%%%%%%%%%%%%%%%%%%
\section*{8. Como Resolver Conflitos Técnicos}

Conflitos são naturais e saudáveis quando bem administrados.

Método recomendado:

\begin{enumerate}
	\item definir o problema com clareza;
	\item listar soluções possíveis;
	\item analisar prós e contras objetivamente;
	\item testar protótipo ou mini-experimento;
	\item escolher a solução baseada em dado, não em opinião;
	\item registrar a decisão para referência futura.
\end{enumerate}

%%%%%%%%%%%%%%%%%%%%%%%%%%%%%%%%%%%%%%%%%%%%%%%%%%%%%%%%%%%%%%%%%%%%
\section*{9. Reuniões de Alinhamento}

Sugestão semanal (15–20 min):

\begin{enumerate}
	\item o que fizemos desde a última reunião;
	\item problemas encontrados;
	\item riscos emergentes;
	\item tarefas para a semana;
	\item redistribuição de responsabilidades;
	\item atualizações no quadro Kanban.
\end{enumerate}

Evitar:

\begin{itemize}
	\item discussões longas;
	\item debates pessoais;
	\item detalhamento excessivo durante a reunião.
\end{itemize}

%%%%%%%%%%%%%%%%%%%%%%%%%%%%%%%%%%%%%%%%%%%%%%%%%%%%%%%%%%%%%%%%%%%%
\section*{10. Motivação, Engajamento e Clima de Trabalho}

Para uma turma de iniciantes, manter motivação é essencial.

Boas práticas:

\begin{itemize}
	\item celebrar entregas semanais;
	\item dividir o trabalho de forma justa;
	\item ajudar quem está atrasado ao invés de criticar;
	\item demonstrar progresso visual no Kanban;
	\item reconhecer esforço individual e coletivo;
	\item evitar comparações entre alunos.
\end{itemize}

Clima positivo reduz evasão e aumenta colaboração.

%%%%%%%%%%%%%%%%%%%%%%%%%%%%%%%%%%%%%%%%%%%%%%%%%%%%%%%%%%%%%%%%%%%%
\section*{11. Checklist do Capítulo}

\begin{itemize}
	\item existem canais claros de comunicação?
	\item todos sabem como registrar tarefas?
	\item a equipe usa branches, PRs e revisões?
	\item reuniões semanais estão acontecendo?
	\item papéis foram definidos e rotacionados?
	\item conflitos técnicos são resolvidos de forma profissional?
	\item decisões estão documentadas?
\end{itemize}

%%%%%%%%%%%%%%%%%%%%%%%%%%%%%%%%%%%%%%%%%%%%%%%%%%%%%%%%%%%%%%%%%%%%
\section*{12. Instruções Passo-a-Passo (Para Leigos)}

\begin{enumerate}
	\item Defina onde cada tipo de mensagem será enviada.
	\item Crie quadro Kanban e distribua tarefas.
	\item Divida papéis (líder técnico, documentador etc.).
	\item Combine regras simples de convivência no projeto.
	\item Registre decisões importantes por escrito.
	\item Revise código do colega antes de aprovar.
	\item Faça reuniões curtas toda semana.
\end{enumerate}

%%%%%%%%%%%%%%%%%%%%%%%%%%%%%%%%%%%%%%%%%%%%%%%%%%%%%%%%%%%%%%%%%%%%
\section*{13. Encerramento do Capítulo}

Boa comunicação transforma um grupo disperso em uma equipe real.  
Boa colaboração transforma alunos iniciantes em desenvolvedores confiantes.  
Boa governança transforma caos em organização.  
Com esses princípios, mesmo uma turma grande é capaz de produzir software de qualidade.
