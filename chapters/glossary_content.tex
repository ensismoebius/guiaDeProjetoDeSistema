\begin{description}
\item[frontend] Camada de interface visível ao usuário, responsável pela apresentação e interação (ex.: HTML, CSS, JavaScript)
\item[backend] Camada de lógica e processamento no servidor, responsável por regras de negócio, armazenamento e API
\item[deploy] Processo de colocar o sistema em produção ou disponibilizá-lo para uso em um ambiente específico
\item[commit] Registro de alteração no repositório de controle de versão contendo um conjunto coerente de modificações
\item[branch] Linha de desenvolvimento no controle de versão, usada para trabalhar de forma isolada em funcionalidades ou correções
\item[endpoint] Ponto de acesso da API que expõe uma funcionalidade específica por meio de uma rota/URL
\item[framework] Estrutura de suporte que provê componentes reutilizáveis e convenções para facilitar o desenvolvimento
\item[pipeline] Fluxo automatizado de etapas (build, testes, deploy) que garante repetibilidade e qualidade nas entregas
\item[release] Versão oficial do software preparada para distribuição ou entrega, geralmente numerada (ex.: v1.0)
\item[testing] Prática de verificar a funcionalidade do sistema por meio de testes manuais ou automatizados
\item[issue] Registro de tarefa, bug ou solicitação de melhoria em ferramentas de rastreamento (ex.: GitHub Issues)
\item[bug] Erro ou falha em um software que causa um comportamento inesperado ou incorreto.
\item[user story] Descrição curta e centrada no usuário de um requisito funcional: formato comum "Como <usuário>, quero <ação> para <benefício>"
\item[Kanban] Método visual para gerenciar fluxo de trabalho por meio de cartões e colunas (ex.: To Do, Doing, Done)
\item[ADR] Architectural Decision Record — documentação que registra decisões arquiteturais importantes e suas justificativas
\item[refactor] Refatoração: processo de reestruturar código sem alterar seu comportamento observável, para melhorar qualidade
\item[CI tooling] Ferramentas e serviços que implementam integração contínua (ex.: GitHub Actions, GitLab CI, Jenkins)
\end{description}