\chapter{Expansão e Aprimoramento do Sistema}
Após o \ac{mvp} estar funcionando, inicia-se a etapa de expansão, onde o sistema
passa a receber melhorias, novas funcionalidades, ajustes mais refinados e
evoluções que o aproximam de um produto final.  
Nesta fase, os alunos aprendem a transformar algo básico em algo mais robusto,
organizado e preparado para uso real.

Para iniciantes, esta etapa é essencial para desenvolver senso de qualidade,
organização de código e responsabilidade técnica.

%%%%%%%%%%%%%%%%%%%%%%%%%%%%%%%%%%%%%%%%%%%%%%%%%%%%%%%%%%%%%%%%%%%%
\section*{1. O Propósito da Expansão}

A expansão existe para:

\begin{itemize}
	\item adicionar funcionalidades além do essencial;
	\item melhorar o que já existe;
	\item corrigir falhas estruturais criadas no \ac{mvp};
	\item aumentar segurança e estabilidade;
	\item preparar o sistema para testes completos.
\end{itemize}

A fase de expansão deve ser feita com calma e foco, porque o código já existe e
qualquer alteração impacta o restante do sistema.

%%%%%%%%%%%%%%%%%%%%%%%%%%%%%%%%%%%%%%%%%%%%%%%%%%%%%%%%%%%%%%%%%%%%
\section*{2. Identificação de Pontos de Melhoria}

Antes de adicionar novidades, é essencial identificar o que precisa ser
aprimorado no \ac{mvp}.  
Os alunos devem observar:

\begin{itemize}
	\item funcionalidades que funcionam, mas “capengam”;
	\item fluxos com muitos cliques desnecessários;
	\item telas confusas ou difíceis de usar;
	\item dados inconsistentes no banco;
	\item problemas de desempenho;
	\item trechos de código repetidos.
\end{itemize}

Um bom exercício é fazer o professor e colegas utilizarem o \ac{mvp} “como usuários”, anotando dificuldades.

%%%%%%%%%%%%%%%%%%%%%%%%%%%%%%%%%%%%%%%%%%%%%%%%%%%%%%%%%%%%%%%%%%%%
\section*{3. Ampliação de Funcionalidades}

Exemplos típicos de expansão:

\begin{itemize}
	\item filtros de pesquisa;
	\item telas com mais detalhes;
	\item validações mais completas no \gls{backend};
	\item melhorias visuais no \gls{frontend};
	\item relatórios simples;
	\item exportação de dados;
	\item histórico de ações;
	\item pequenas automações internas.
\end{itemize}

Os alunos precisam aprender que novidades vêm \textbf{depois} que o básico está sólido.

%%%%%%%%%%%%%%%%%%%%%%%%%%%%%%%%%%%%%%%%%%%%%%%%%%%%%%%%%%%%%%%%%%%%
\section*{4. Refino do Código Existente}

A expansão não significa apenas adicionar coisas novas, mas melhorar o que já
existe. Para iniciantes, isso inclui:

\begin{itemize}
	\item renomear funções para deixá-las mais claras;
	\item dividir arquivos grandes em módulos menores;
	\item extrair funções reaproveitáveis;
	\item remover duplicações;
	\item limpar comentários desnecessários;
	\item padronizar nomes de variáveis e pastas.
\end{itemize}

Refinar código é essencial para evitar um sistema difícil de manter.

%%%%%%%%%%%%%%%%%%%%%%%%%%%%%%%%%%%%%%%%%%%%%%%%%%%%%%%%%%%%%%%%%%%%
\section*{5. Aprimoramento Visual das Telas}

Uma interface mais organizada facilita o uso e reduz confusão.

Melhorias comuns:

\begin{itemize}
	\item espaçamento mais consistente;
	\item alinhamento adequado;
	\item botões mais visíveis;
	\item cores coerentes;
	\item feedback visual para ações (mensagens de sucesso/erro);
	\item revisões de acessibilidade.
\end{itemize}

Pequenas melhorias de visualização aumentam a motivação dos alunos ao ver o sistema ficar “bonito”.

%%%%%%%%%%%%%%%%%%%%%%%%%%%%%%%%%%%%%%%%%%%%%%%%%%%%%%%%%%%%%%%%%%%%
\section*{6. Ampliação do Banco de Dados}

À medida que o sistema cresce, o banco precisa acompanhar.  
Possíveis expansões:

\begin{itemize}
	\item criação de novas tabelas;
	\item adição de novos campos;
	\item criação de relacionamentos mais complexos;
	\item implementação de chaves estrangeiras;
	\item inclusão de índices para acelerar buscas.
\end{itemize}

Cada alteração deve ser registrada e comunicada ao time para evitar inconsistências.

%%%%%%%%%%%%%%%%%%%%%%%%%%%%%%%%%%%%%%%%%%%%%%%%%%%%%%%%%%%%%%%%%%%%
\section*{7. Melhoria do \gls{backend}}

O \gls{backend} cresce em complexidade com a expansão.  
Boas práticas:

\begin{itemize}
	\item adicionar logs simples;
	\item melhorar o tratamento de erros;
	\item padronizar mensagens de retorno;
	\item criar funções auxiliares para evitar duplicação;
	\item revisar tratamentos de segurança.
\end{itemize}

Evitar deixar o \gls{backend} virar um amontoado de rotas improvisadas é uma habilidade essencial para iniciantes.

%%%%%%%%%%%%%%%%%%%%%%%%%%%%%%%%%%%%%%%%%%%%%%%%%%%%%%%%%%%%%%%%%%%%
\section*{8. Melhoria do \gls{frontend}}

No \gls{frontend}, melhorias típicas incluem:

\begin{itemize}
	\item navegação mais clara;
	\item formulários com validação visual;
	\item feedback imediato;
	\item componentes reaproveitáveis;
	\item mensagens de erro informativas;
	\item loading indicators.
\end{itemize}

O aluno deve aprender que uma tela realmente boa evita erros do usuário.

%%%%%%%%%%%%%%%%%%%%%%%%%%%%%%%%%%%%%%%%%%%%%%%%%%%%%%%%%%%%%%%%%%%%
\section*{9. Revisão de Fluxos de Uso}

Nesta fase, o grupo deve revisar cada fluxo de uso do sistema:

\begin{enumerate}
	\item cadastro;
	\item edição;
	\item listagem;
	\item exclusão;
	\item login;
	\item logout;
\end{enumerate}

Verificar se:

\begin{itemize}
	\item existem cliques desnecessários;
	\item há informações faltando;
	\item a navegação é intuitiva;
	\item o usuário entende o que está acontecendo.
\end{itemize}

%%%%%%%%%%%%%%%%%%%%%%%%%%%%%%%%%%%%%%%%%%%%%%%%%%%%%%%%%%%%%%%%%%%%
\section*{10. Checklist da Expansão}

\begin{itemize}
	\item \ac{mvp} funcionando sem erros graves.
	\item novas funcionalidades pequenas e bem definidas;
	\item melhorias visuais aplicadas;
	\item refatorações feitas com cuidado;
	\item banco ajustado conforme necessidade;
	\item \gls{backend} mais organizado;
	\item \gls{frontend} mais intuitivo.
\end{itemize}

%%%%%%%%%%%%%%%%%%%%%%%%%%%%%%%%%%%%%%%%%%%%%%%%%%%%%%%%%%%%%%%%%%%%
\section*{11. Instruções Passo-a-Passo (Para Leigos)}

\begin{enumerate}
	\item Use o \ac{mvp} e procure falhas e pontos fracos.
	\item Liste tudo o que pode ser melhorado.
	\item Comece pelas melhorias pequenas.
	\item Depois adicione funcionalidades novas.
	\item Teste cada melhoria imediatamente.
	\item Se quebrar algo, volte ao que funcionava antes.
	\item Organize o código para ficar mais limpo.
	\item Revise o banco e ajuste tabelas e relacionamentos.
\end{enumerate}

%%%%%%%%%%%%%%%%%%%%%%%%%%%%%%%%%%%%%%%%%%%%%%%%%%%%%%%%%%%%%%%%%%%%
\section*{12. Encerramento do Capítulo}

A expansão é a etapa onde o sistema começa a ganhar maturidade.  
Os alunos aprendem a pensar como desenvolvedores:  
melhorar, refinar, organizar e evoluir o código com responsabilidade.  
É nesta fase que a visão de um produto real começa a surgir.
