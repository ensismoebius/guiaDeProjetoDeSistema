\chapter{Testes e Controle de Qualidade}
Garantir a qualidade de um software é parte fundamental do processo de
desenvolvimento.  
Testes não servem apenas para "procurar erros", mas para confirmar que o
sistema está funcionando como deveria e que continuará funcionando conforme
crescer.  
Para alunos iniciantes, esta etapa é essencial para aprender boas práticas,
evitar bugs e ganhar confiança no próprio código.

%%%%%%%%%%%%%%%%%%%%%%%%%%%%%%%%%%%%%%%%%%%%%%%%%%%%%%%%%%%%%%%%%%%%
\section*{1. O Que é Qualidade em Software?}

Qualidade significa que o sistema:

\begin{itemize}
	\item funciona corretamente;
	\item é fácil de usar;
	\item é estável;
	\item é coerente;
	\item é seguro;
	\item é compreensível.
\end{itemize}

A qualidade não aparece por acaso: é construída com disciplina e testes
frequentes.

%%%%%%%%%%%%%%%%%%%%%%%%%%%%%%%%%%%%%%%%%%%%%%%%%%%%%%%%%%%%%%%%%%%%
\section*{2. Tipos de Testes Explicados de Forma Simples}

\subsection*{2.1 Testes Manuais}
O aluno executa o sistema e verifica se a funcionalidade funciona conforme o esperado.

Exemplo:
\begin{itemize}
	\item clicar no botão “Salvar” deve salvar o item;
	\item clicar em “Excluir” deve remover da lista;
	\item tentar salvar algo sem preencher campos obrigatórios deve dar erro.
\end{itemize}

\subsection*{2.2 Testes Automatizados}
São scripts de teste que verificam funcionalidades automaticamente.

Dividem-se em:

\begin{itemize}
	\item \textbf{Testes unitários}: testam funções isoladas;
	\item \textbf{Testes de integração}: verificam partes que trabalham juntas;
	\item \textbf{Testes de ponta a ponta}: simulam todo o fluxo do usuário.
\end{itemize}

\subsection*{2.3 Testes de Usabilidade}
Avaliam se o sistema é simples e intuitivo para usuários reais.

%%%%%%%%%%%%%%%%%%%%%%%%%%%%%%%%%%%%%%%%%%%%%%%%%%%%%%%%%%%%%%%%%%%%
\section*{3. Por Que Iniciantes Devem Testar?}

\begin{itemize}
	\item reduz erros inesperados;
	\item evita retrabalho;
	\item melhora a qualidade do código;
	\item reforça compreensão do funcionamento interno;
	\item aumenta a confiança no próprio desenvolvimento.
\end{itemize}

Quanto mais cedo o aluno testa, menos problemas ele encontra no final.

%%%%%%%%%%%%%%%%%%%%%%%%%%%%%%%%%%%%%%%%%%%%%%%%%%%%%%%%%%%%%%%%%%%%
\section*{4. Roteiros de Teste}

Um roteiro de teste é uma lista organizada de itens a serem verificados.

Exemplo simples:

\begin{itemize}
	\item abrir tela de cadastro;
	\item inserir dados válidos;
	\item clicar em salvar;
	\item verificar se aparece na listagem;
	\item editar o item;
	\item excluir o item.
\end{itemize}

Cada item deve ter:

\begin{itemize}
	\item ação a realizar;
	\item resultado esperado;
	\item resultado obtido.
\end{itemize}

%%%%%%%%%%%%%%%%%%%%%%%%%%%%%%%%%%%%%%%%%%%%%%%%%%%%%%%%%%%%%%%%%%%%
\section*{5. Boas Práticas de Testes para Iniciantes}

\begin{itemize}
	\item testar pequenas partes do sistema frequentemente;
	\item testar cada funcionalidade logo após implementá-la;
	\item preparar dados simples para facilitar a validação;
	\item registrar erros encontrados;
	\item refazer o teste após corrigir o erro;
	\item revisar com o professor quando necessário.
\end{itemize}

%%%%%%%%%%%%%%%%%%%%%%%%%%%%%%%%%%%%%%%%%%%%%%%%%%%%%%%%%%%%%%%%%%%%
\section*{6. Testando o Frontend}

Elementos a verificar:

\begin{itemize}
	\item botões clicam corretamente;
	\item campos validam dados;
	\item mensagens de erro são claras;
	\item o layout não quebra;
	\item a navegação é intuitiva.
\end{itemize}

%%%%%%%%%%%%%%%%%%%%%%%%%%%%%%%%%%%%%%%%%%%%%%%%%%%%%%%%%%%%%%%%%%%%
\section*{7. Testando o Backend}

Itens a verificar:

\begin{itemize}
	\item rotas respondem corretamente;
	\item validações tratam entradas inválidas;
	\item dados são gravados corretamente no banco;
	\item erros são tratados sem travar o sistema;
	\item respostas têm formato consistente.
\end{itemize}

%%%%%%%%%%%%%%%%%%%%%%%%%%%%%%%%%%%%%%%%%%%%%%%%%%%%%%%%%%%%%%%%%%%%
\section*{8. Testando o Banco de Dados}

Verificar:

\begin{itemize}
	\item tabelas criadas corretamente;
	\item relacionamentos funcionando como o esperado;
	\item chaves primárias e estrangeiras aplicadas;
	\item registros válidos e consistentes;
	\item dados duplicados não sendo aceitos sem controle.
\end{itemize}

%%%%%%%%%%%%%%%%%%%%%%%%%%%%%%%%%%%%%%%%%%%%%%%%%%%%%%%%%%%%%%%%%%%%
\section*{9. Checklist Completo de Qualidade}

\begin{itemize}
	\item cada tela testada;
	\item cada rota verificada;
	\item banco funcionando sem inconsistências;
	\item todos os fluxos principais revisados;
	\item erros reportados e corrigidos;
	\item sistema estável após correções.
\end{itemize}

%%%%%%%%%%%%%%%%%%%%%%%%%%%%%%%%%%%%%%%%%%%%%%%%%%%%%%%%%%%%%%%%%%%%
\section*{10. Estratégia Simples de Controle de Qualidade}

Sugestão didática:

\begin{enumerate}
	\item alunos desenvolvem durante a semana;
	\item no início da aula seguinte, cada grupo testa o sistema de outro grupo;
	\item criam lista de problemas encontrados;
	\item voltam para corrigir seus próprios erros;
	\item repetem o ciclo.
\end{enumerate}

Essa estratégia aumenta:

\begin{itemize}
	\item senso de responsabilidade;
	\item visão crítica;
	\item qualidade geral do projeto;
	\item interação entre os alunos.
\end{itemize}

%%%%%%%%%%%%%%%%%%%%%%%%%%%%%%%%%%%%%%%%%%%%%%%%%%%%%%%%%%%%%%%%%%%%
\section*{11. Erros Comuns em Testes (e Como Evitar)}

\begin{itemize}
	\item \textbf{não testar pequenos trechos}: causa acúmulo de erros;
	\item \textbf{testar apenas “de vez em quando”}: dificulta encontrar causas;
	\item \textbf{testar apenas o que funciona}: ignora falhas importantes;
	\item \textbf{não anotar resultados}: dificulta correções futuras;
	\item \textbf{pressa para “terminar logo”}: leva a problemas maiores depois.
\end{itemize}

%%%%%%%%%%%%%%%%%%%%%%%%%%%%%%%%%%%%%%%%%%%%%%%%%%%%%%%%%%%%%%%%%%%%
\section*{12. Instruções Passo-a-Passo (Para Leigos)}

\begin{enumerate}
	\item Abra o sistema e escolha uma funcionalidade.
	\item Realize cada ação lentamente.
	\item Observe o que acontece.
	\item Compare com o que deveria acontecer.
	\item Anote qualquer diferença.
	\item Após corrigir, repita o teste.
	\item Teste mais uma vez para garantir.
\end{enumerate}

%%%%%%%%%%%%%%%%%%%%%%%%%%%%%%%%%%%%%%%%%%%%%%%%%%%%%%%%%%%%%%%%%%%%
\section*{13. Encerramento do Capítulo}

Testar é aprender.  
Ao testar, o aluno vê claramente como o código funciona na prática, identifica
erros que não percebia e compreende melhor o sistema como um todo.  
Um projeto de software só é realmente confiável quando passou por testes
sistemáticos e controle de qualidade adequado.
