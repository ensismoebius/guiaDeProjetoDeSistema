\chapter{Gestão de Continuidade, Evolução do Software e Boas Práticas de Pós-Projeto}

Mesmo após a entrega final, um software bem construído pode (e deve) continuar
evoluindo. Este capítulo explica como manter, estender, revisar e melhorar o
projeto após o encerramento do semestre, seguindo práticas adotadas em projetos
profissionais, porém apresentadas de forma acessível para iniciantes.

A continuidade é um dos fatores que mais diferenciam sistemas amadores de
sistemas robustos. Quando um time aprende a manter software após a entrega,
desenvolve maturidade técnica e disciplina.

%%%%%%%%%%%%%%%%%%%%%%%%%%%%%%%%%%%%%%%%%%%%%%%%%%%%%%%%%%%%%%%%%%%%
\section*{1. O Que Significa Evoluir um Software?}

Evoluir um sistema significa:

\begin{itemize}
	\item corrigir problemas descobertos após o uso;
	\item adicionar melhorias solicitadas por usuários;
	\item refatorar partes complexas;
	\item melhorar desempenho;
	\item simplificar fluxos confusos;
	\item adaptar funcionalidades para novos cenários.
\end{itemize}

Para alunos, essa fase é a oportunidade ideal de:

\begin{itemize}
	\item reforçar conceitos de engenharia;
	\item praticar manutenção de código;
	\item aprender refatoração gradual;
	\item trabalhar com versões, branches e releases.
\end{itemize}

%%%%%%%%%%%%%%%%%%%%%%%%%%%%%%%%%%%%%%%%%%%%%%%%%%%%%%%%%%%%%%%%%%%%
\section*{2. Modelos de Evolução Usados na Indústria}

Mesmo iniciantes podem adotar modelos simplificados de evolução:

\subsection*{2.1 Manutenção Corretiva}
Conserta falhas e bugs encontrados após o uso.

\subsection*{2.2 Manutenção Adaptativa}
Modifica o sistema para funcionar em novos ambientes, versões de biblioteca ou sistemas operacionais.

\subsection*{2.3 Manutenção Perfectiva}
Aprimora funções existentes, reorganiza código e melhora desempenho.

\subsection*{2.4 Manutenção Preventiva}
Evita que erros futuros ocorram, reforçando estrutura e padrões.

%%%%%%%%%%%%%%%%%%%%%%%%%%%%%%%%%%%%%%%%%%%%%%%%%%%%%%%%%%%%%%%%%%%%
\section*{3. Planejamento da Continuidade}

Para continuar um projeto após o semestre, o time deve planejar:

\begin{itemize}
	\item lista de melhorias desejadas (backlog);
	\item priorização das tarefas;
	\item organização dos papéis dos integrantes;
	\item ciclos curtos de evolução;
	\item metas mensais claras e realistas.
\end{itemize}

Mesmo com alunos iniciantes, é possível manter um ritmo saudável:

\begin{itemize}
	\item 1 melhoria simples por semana;
	\item 1 revisão de código;
	\item 1 teste manual completo;
	\item checkpoints pontuais.
\end{itemize}

%%%%%%%%%%%%%%%%%%%%%%%%%%%%%%%%%%%%%%%%%%%%%%%%%%%%%%%%%%%%%%%%%%%%
\section*{4. Organização do Repositório Após o Semestre}

Uma boa estrutura facilita a continuidade:

\begin{itemize}
	\item branches antigas arquivadas;
	\item documentação atualizada;
	\item pastas reorganizadas;
	\item commits limpos e descritivos;
	\item roteiro de instalação revisado.
\end{itemize}

Sugestão simples de organização:

\begin{verbatim}
	main → versão estável
	dev → desenvolvimento contínuo
	archive → código antigo
\end{verbatim}

%%%%%%%%%%%%%%%%%%%%%%%%%%%%%%%%%%%%%%%%%%%%%%%%%%%%%%%%%%%%%%%%%%%%
\section*{5. Ferramentas para Evolução Gradual}

Recursos acessíveis para continuar o desenvolvimento:

\begin{itemize}
	\item issues para gerenciar novas tarefas;
	\item milestones para agrupar entregas;
	\item pull requests para revisão colaborativa;
	\item tags para marcar versões;
	\item wiki para documentações recorrentes.
\end{itemize}

Essas ferramentas são fáceis de usar e treinam disciplina técnica.

%%%%%%%%%%%%%%%%%%%%%%%%%%%%%%%%%%%%%%%%%%%%%%%%%%%%%%%%%%%%%%%%%%%%
\section*{6. Comunicação e Alinhamento do Time}

A evolução eficaz depende de comunicação clara.

Práticas recomendadas:

\begin{itemize}
	\item reuniões curtas mensais;
	\item registro formal de decisões;
	\item documentação viva atualizada;
	\item comentários claros nas issues.
\end{itemize}

Erros comuns a evitar:

\begin{itemize}
	\item assumir que todos sabem o que deve ser feito;
	\item começar melhorias sem avisar o time;
	\item misturar código experimental na versão principal.
\end{itemize}

%%%%%%%%%%%%%%%%%%%%%%%%%%%%%%%%%%%%%%%%%%%%%%%%%%%%%%%%%%%%%%%%%%%%
\section*{7. Obstáculos Comuns na Continuidade}

Ao longo dos anos, observam-se padrões típicos em turmas iniciantes:

\begin{itemize}
	\item desmotivação pós-semestre;
	\item dificuldade em entender código próprio;
	\item falta de documentação histórica;
	\item branches abandonadas;
	\item dependências quebradas após atualizações.
\end{itemize}

A solução é adotar:

\begin{itemize}
	\item pequenas melhorias;
	\item revisões periódicas;
	\item padronização mínima;
	\item checkpoints simples.
\end{itemize}

%%%%%%%%%%%%%%%%%%%%%%%%%%%%%%%%%%%%%%%%%%%%%%%%%%%%%%%%%%%%%%%%%%%%
\section*{8. Boas Práticas de Código para Facilitar Evolução}

Equipes iniciantes devem aplicar:

\begin{itemize}
	\item funções curtas;
	\item nomes claros de variáveis;
	\item comentários explicando “por que”, não só “o que”;
	\item uso consistente de padrões;
	\item modularização progressiva.
\end{itemize}

Refatorar não é reescrever: é melhorar aos poucos.

%%%%%%%%%%%%%%%%%%%%%%%%%%%%%%%%%%%%%%%%%%%%%%%%%%%%%%%%%%%%%%%%%%%%
\section*{9. Como Evitar Dívida Técnica}

Dívida técnica ocorre quando decisões rápidas criam problemas futuros.

Para minimizar:

\begin{itemize}
	\item evitar atalhos temporários;
	\item revisar código antigo periodicamente;
	\item corrigir warnings
	\item remover duplicações;
	\item documentar o que será melhorado no futuro;
	\item padronizar lógicas repetidas.
\end{itemize}

%%%%%%%%%%%%%%%%%%%%%%%%%%%%%%%%%%%%%%%%%%%%%%%%%%%%%%%%%%%%%%%%%%%%
\section*{10. Continuidade Educacional: Transformando o Projeto em Aprendizado}

A evolução pós-projeto permite ao aluno:

\begin{itemize}
	\item praticar técnicas avançadas;
	\item melhorar qualidade do portfólio;
	\item entender engenharia de software de verdade;
	\item consolidar autonomia técnica.
\end{itemize}

Sugestões:

\begin{itemize}
	\item migrar parte do sistema para outra linguagem;
	\item criar protótipos de novas versões;
	\item escrever testes automatizados extras;
	\item convidar colegas de outros semestres para participar.
\end{itemize}

%%%%%%%%%%%%%%%%%%%%%%%%%%%%%%%%%%%%%%%%%%%%%%%%%%%%%%%%%%%%%%%%%%%%
\section*{11. Estratégias Motivacionais Baseadas em Evidências}

Pesquisas educacionais mostram que alunos mantêm projetos por mais tempo quando:

\begin{itemize}
	\item sentem orgulho do que construíram;
	\item tem um propósito além da nota;
	\item trabalham em grupo pequeno e coeso;
	\item enxergam melhoria contínua;
	\item recebem reconhecimento público.
\end{itemize}

Estratégias práticas:

\begin{itemize}
	\item publicar releases em redes sociais;
	\item apresentar melhorias para novas turmas;
	\item registrar evolução com prints antes/depois;
	\item manter um changelog simples.
\end{itemize}

%%%%%%%%%%%%%%%%%%%%%%%%%%%%%%%%%%%%%%%%%%%%%%%%%%%%%%%%%%%%%%%%%%%%
\section*{12. Checklist do Capítulo}

\begin{itemize}
	\item há backlog organizado para evolução?
	\item documentação está atualizada?
	\item versões antigas foram arquivadas?
	\item bugs não resolvidos estão registrados?
	\item melhorias estão priorizadas?
	\item equipe sabe quem faz o quê?
\end{itemize}

%%%%%%%%%%%%%%%%%%%%%%%%%%%%%%%%%%%%%%%%%%%%%%%%%%%%%%%%%%%%%%%%%%%%
\section*{13. Instruções Passo-a-Passo (Para Leigos)}

\begin{enumerate}
	\item abra o repositório do projeto.
	\item liste problemas e melhorias desejadas.
	\item crie issues para cada item.
	\item escolha uma melhoria simples.
	\item crie um branch.
	\item faça a alteração.
	\item teste o sistema inteiro.
	\item registre o que mudou no changelog.
	\item envie um pull request.
	\item repita mensalmente.
\end{enumerate}

%%%%%%%%%%%%%%%%%%%%%%%%%%%%%%%%%%%%%%%%%%%%%%%%%%%%%%%%%%%%%%%%%%%%
\section*{14. Encerramento do Capítulo}

A continuidade e evolução são a ponte entre o ambiente acadêmico e o
profissional.  
Quando o aluno aprende a manter o que construiu, desenvolve responsabilidade,
conhecimento duradouro e independência técnica.  
O projeto não termina na entrega: ele se transforma em oportunidade de
crescimento contínuo.
