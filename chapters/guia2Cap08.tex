\chapter{Documentação e Entrega Final}
A documentação é a memória do projeto.  
Ela registra o que foi criado, como funciona, como usar e como manter.  
Para iniciantes, documentar significa organizar o próprio pensamento e mostrar
clareza sobre o que foi feito.

A entrega final reúne todos os materiais do projeto em sua forma mais completa:
código consolidado, documentação organizada, demonstração funcional e relatório
final.  
Nesta fase, o grupo aprende a apresentar seu trabalho de forma profissional.

%%%%%%%%%%%%%%%%%%%%%%%%%%%%%%%%%%%%%%%%%%%%%%%%%%%%%%%%%%%%%%%%%%%%
\section*{1. A Importância da Documentação}

Documentar é tão importante quanto programar.  
Sem documentação, ninguém sabe:

\begin{itemize}
	\item como o sistema funciona;
	\item como instalar ou rodar;
	\item quais são as regras de negócio;
	\item onde alterar o código;
	\item quais decisões foram tomadas;
	\item o que já foi testado;
	\item o que ainda precisa ser feito.
\end{itemize}

Para iniciantes, documentar ajuda a fixar aprendizado e criar consciência de
qualidade.

%%%%%%%%%%%%%%%%%%%%%%%%%%%%%%%%%%%%%%%%%%%%%%%%%%%%%%%%%%%%%%%%%%%%
\section*{2. Tipos de Documentação Necessários}

A documentação final deve incluir:

\subsection*{2.1 Manual Técnico}
Voltado ao programador.  
Inclui:

\begin{itemize}
	\item arquitetura do sistema;
	\item estrutura de pastas e arquivos;
	\item explicação de módulos e funções principais;
	\item rotas e endpoints do \gls{backend};
	\item modelo do banco de dados;
	\item pré-requisitos técnicos.
\end{itemize}

\subsection*{2.2 Manual do Usuário}
Voltado ao público não técnico.  
Deve conter:

\begin{itemize}
	\item telas do sistema;
	\item instruções de uso passo-a-passo;
	\item exemplos com dados reais;
	\item como resolver erros comuns.
\end{itemize}

\subsection*{2.3 Guia de Instalação}
Simples e objetivo:

\begin{itemize}
	\item tecnologias necessárias;
	\item comandos de instalação;
	\item como iniciar o servidor;
	\item como acessar o sistema.
\end{itemize}

\subsection*{2.4 Relatório Final}
Documento que explica o projeto como um todo:

\begin{itemize}
	\item problema inicial;
	\item requisitos;
	\item arquitetura;
	\item modelo de dados;
	\item funcionalidades implementadas;
	\item funcionalidades não implementadas e motivos;
	\item riscos enfrentados;
	\item lições aprendidas pelo grupo.
\end{itemize}

%%%%%%%%%%%%%%%%%%%%%%%%%%%%%%%%%%%%%%%%%%%%%%%%%%%%%%%%%%%%%%%%%%%%
\section*{3. Ferramentas Simples para Documentação}

Para alunos iniciantes:

\begin{itemize}
	\item Markdown (GitHub)
	\item PDF gerado a partir de editor de texto
	\item Google Docs
	\item Arquivos .md no repositório
\end{itemize}

O importante não é o formato, mas a clareza.

%%%%%%%%%%%%%%%%%%%%%%%%%%%%%%%%%%%%%%%%%%%%%%%%%%%%%%%%%%%%%%%%%%%%
\section*{4. Organização da Estrutura do Projeto}

Recomenda-se a seguinte estrutura de pastas para o repositório final:

\begin{verbatim}
	/
	|-- src/
	|   |-- backend/
	|   |-- frontend/
	|   \-- database/
	|-- docs/
	|   |-- manual-tecnico.md
	|   |-- manual-usuario.md
	|   |-- guia-instalacao.md
	|   \-- relatorio-final.md
	|-- tests/
	\-- README.md
\end{verbatim}

Essa estrutura ajuda tanto alunos quanto avaliadores a entenderem o projeto.

%%%%%%%%%%%%%%%%%%%%%%%%%%%%%%%%%%%%%%%%%%%%%%%%%%%%%%%%%%%%%%%%%%%%
\section*{5. Como Fazer Prints de Tela de Forma Profissional}

Para iniciantes:

\begin{itemize}
	\item usar fundo limpo;
	\item não incluir abas desnecessárias do navegador;
	\item usar resolução mínima de 1280px;
	\item focar apenas na funcionalidade demonstrada;
	\item adicionar legendas curtas e objetivas;
	\item numerar as telas quando fizer sentido.
\end{itemize}

%%%%%%%%%%%%%%%%%%%%%%%%%%%%%%%%%%%%%%%%%%%%%%%%%%%%%%%%%%%%%%%%%%%%
\section*{6. A Demonstração Final do Sistema}

A apresentação final deve incluir:

\begin{enumerate}
	\item breve explicação do problema resolvido;
	\item tecnologias utilizadas;
	\item demonstração do fluxo principal;
	\item explicação das dificuldades enfrentadas;
	\item lições aprendidas;
	\item evolução do \ac{mvp} até a versão final.
\end{enumerate}

A demonstração deve ser clara, simples e objetiva.

%%%%%%%%%%%%%%%%%%%%%%%%%%%%%%%%%%%%%%%%%%%%%%%%%%%%%%%%%%%%%%%%%%%%
\section*{7. Ensaiando a Apresentação}

Antes da apresentação oficial:

\begin{itemize}
	\item testar o sistema na máquina que será usada na apresentação;
	\item verificar conexão com internet (se necessária);
	\item organizar quem fala o quê;
	\item treinar a fala para não ultrapassar o tempo;
	\item preparar um plano B (vídeo curto ou prints).
\end{itemize}

Para iniciantes, ensaio reduz ansiedade e aumenta segurança.

%%%%%%%%%%%%%%%%%%%%%%%%%%%%%%%%%%%%%%%%%%%%%%%%%%%%%%%%%%%%%%%%%%%%
\section*{8. Checklist da Documentação Final}

\begin{itemize}
	\item manual técnico completo;
	\item manual do usuário revisado;
	\item guia de instalação testado;
	\item relatório final organizado;
	\item código final limpo e comentado;
	\item prints de tela consistentes;
	\item repositório organizado;
	\item apresentação ensaiada.
\end{itemize}

%%%%%%%%%%%%%%%%%%%%%%%%%%%%%%%%%%%%%%%%%%%%%%%%%%%%%%%%%%%%%%%%%%%%
\section*{9. Instruções Passo-a-Passo (Para Leigos)}

\begin{enumerate}
	\item Abra a pasta do projeto.
	\item Crie uma pasta chamada “docs”.
	\item Crie quatro arquivos: manual técnico, manual do usuário, guia de instalação e relatório final.
	\item Tire prints das telas e coloque no documento adequado.
	\item Escreva tudo de forma simples.
	\item Teste o guia de instalação em outro computador.
	\item Organize a apresentação do sistema.
\end{enumerate}

%%%%%%%%%%%%%%%%%%%%%%%%%%%%%%%%%%%%%%%%%%%%%%%%%%%%%%%%%%%%%%%%%%%%
\section*{10. Encerramento do Capítulo}

A documentação e a entrega final transformam o projeto em algo completo,
profissional e compreensível.  
Para os alunos, esta etapa marca a conclusão da jornada, mostrando que são
capazes de criar, organizar, apresentar e explicar um sistema funcional do
início ao fim.  
Mais do que entregar o software, eles entregam conhecimento consolidado.
