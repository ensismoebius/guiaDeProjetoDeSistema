\chapter{Definição do Problema e Requisitos}
Este capítulo descreve como transformar uma ideia inicial em um conjunto 
organizado de requisitos. Esta etapa determina o que será construído, 
qual é o objetivo do software e quais funcionalidades são realmente 
necessárias. Para iniciantes, esta é a etapa que mais reduz erros futuros.

%%%%%%%%%%%%%%%%%%%%%%%%%%%%%%%%%%%%%%%%%%%%%%%%%%%%%%%%%%%%%%%%%%%%
\section*{1. O que é “Definir o Problema”?}

Definir o problema significa identificar:

\begin{itemize}
	\item quem precisa do software;
	\item o que essa pessoa deseja resolver;
	\item por que o problema existe;
	\item qual é a solução esperada;
	\item com quais limitações o projeto deve conviver.
\end{itemize}

Para alunos iniciantes, esse passo evita que o projeto vire apenas um conjunto
desordenado de códigos e funções desconectadas. Uma boa definição do problema
funciona como um mapa que guia todo o restante do projeto.

%%%%%%%%%%%%%%%%%%%%%%%%%%%%%%%%%%%%%%%%%%%%%%%%%%%%%%%%%%%%%%%%%%%%
\section*{2. Como Identificar o Problema Central}

Todo software existe para resolver um problema real. Os alunos precisam aprender 
a reconhecer esse problema central. Uma forma simples é responder:

\begin{itemize}
	\item \textbf{O que está dando errado hoje?}
	\item \textbf{Quem sofre com isso?}
	\item \textbf{O que seria uma solução ideal?}
\end{itemize}

Exemplos didáticos:

\textbf{Problema}: clientes não conseguem acompanhar pedidos.  
\textbf{Problema}: funcionários não conseguem organizar horários.  
\textbf{Problema}: dados são perdidos porque não há registro digital.

A clareza na definição evita que o time construía funcionalidades sem propósito.

%%%%%%%%%%%%%%%%%%%%%%%%%%%%%%%%%%%%%%%%%%%%%%%%%%%%%%%%%%%%%%%%%%%%
\section*{3. Tipos de Requisitos}

Os requisitos são divididos em duas categorias principais:

\subsection*{3.1 Requisitos Funcionais (RF)}
Descrevem \textbf{o que o sistema faz}.  
São funções, ações, comportamentos observáveis.

Exemplos:
\begin{itemize}
	\item RF01: O sistema deve permitir login de usuários.
	\item RF02: O sistema deve registrar pedidos.
	\item RF03: O sistema deve gerar relatórios mensais.
\end{itemize}

\subsection*{3.2 Requisitos Não Funcionais (RNF)}
Descrevem \textbf{como o sistema deve funcionar}.  
São regras de qualidade, desempenho, segurança, confiabilidade, etc.

Exemplos:
\begin{itemize}
	\item RNF01: O sistema deve responder em menos de 2 segundos.
	\item RNF02: A interface deve ser acessível para iniciantes.
	\item RNF03: O sistema deve funcionar em navegadores modernos.
\end{itemize}

%%%%%%%%%%%%%%%%%%%%%%%%%%%%%%%%%%%%%%%%%%%%%%%%%%%%%%%%%%%%%%%%%%%%
\section*{4. Ferramentas de Apoio à Definição de Requisitos}

Como os alunos estão em início de carreira, é importante usar ferramentas simples, 
intuitivas e colaborativas.

Ferramentas sugeridas:

\begin{itemize}
	\item GitHub Issues (para documentar requisitos individualmente);
	\item Google Forms (para entrevistas e coleta de opiniões);
	\item Draw.io ou Excalidraw (para modelos de tela simples);
	\item Google Docs (para rascunhos colaborativos).
\end{itemize}

O objetivo é que todos os alunos consigam participar e visualizar a evolução 
dos requisitos, mesmo com pouca prática.

%%%%%%%%%%%%%%%%%%%%%%%%%%%%%%%%%%%%%%%%%%%%%%%%%%%%%%%%%%%%%%%%%%%%
\section*{5. Como Validar se o Requisito Está Claro}

Um requisito bem definido deve:

\begin{itemize}
	\item ser curto e direto;
	\item não ter ambiguidade;
	\item não depender de interpretações;
	\item ser testável;
	\item descrever apenas uma função;
	\item não misturar solução com necessidade.
\end{itemize}

\textbf{Regra prática}:  
Se dois alunos lerem o mesmo requisito e concluírem coisas diferentes,  
o requisito está mal escrito.

%%%%%%%%%%%%%%%%%%%%%%%%%%%%%%%%%%%%%%%%%%%%%%%%%%%%%%%%%%%%%%%%%%%%
\section*{6. Erros Comuns em Definição de Problemas e Requisitos}

\textbf{Erro}: escrever requisitos vagos, como “melhorar desempenho”.  
\textbf{Correção}: definir valores objetivos (“responder em até 1 segundo”).

\textbf{Erro}: colocar múltiplas funções no mesmo requisito.  
\textbf{Correção}: criar um requisito por funcionalidade.

\textbf{Erro}: escrever requisitos como tutoriais (“o usuário clica no botão…”)  
\textbf{Correção}: requisitos descrevem intenção, não passo-a-passo.

\textbf{Erro}: confundir requisito com interface gráfica.  
\textbf{Correção}: primeiro define-se o que deve existir; depois, como será exibido.

%%%%%%%%%%%%%%%%%%%%%%%%%%%%%%%%%%%%%%%%%%%%%%%%%%%%%%%%%%%%%%%%%%%%
\section*{7. Checklist do Capítulo}

Ao final deste capítulo, os alunos devem conseguir responder:

\begin{itemize}
	\item Quem é o usuário principal?
	\item Qual problema ele enfrenta hoje?
	\item O que seria uma solução satisfatória?
	\item Quais são os requisitos funcionais essenciais?
	\item Quais são os requisitos não funcionais obrigatórios?
	\item Quais requisitos são opcionais?
\end{itemize}

Se a turma não consegue responder a essas perguntas, o projeto ainda não está pronto para seguir.

%%%%%%%%%%%%%%%%%%%%%%%%%%%%%%%%%%%%%%%%%%%%%%%%%%%%%%%%%%%%%%%%%%%%
\section*{8. Instruções Passo-a-Passo (Para Leigos)}

\begin{enumerate}
	\item Pegue um caderno ou documento digital.
	\item Escreva em uma frase simples qual é o problema que o software deve resolver.
	\item Liste quem será beneficiado pelo software.
	\item Escreva três problemas que essa pessoa enfrenta hoje.
	\item Escreva três soluções que resolveriam esses problemas.
	\item Transforme cada solução em um requisito funcional.
	\item Escreva três requisitos não funcionais básicos (desempenho, segurança, usabilidade).
	\item Abra o GitHub.
	\item Crie uma issue para cada requisito.
	\item Dê títulos claros, como “RF01 — Cadastro de usuários”.
	\item Peça para outro colega ler cada requisito e dizer o que entendeu.
	\item Se houver divergência, reescreva até ficar claro.
\end{enumerate}

%%%%%%%%%%%%%%%%%%%%%%%%%%%%%%%%%%%%%%%%%%%%%%%%%%%%%%%%%%%%%%%%%%%%
\section*{9. Encerramento do Capítulo}

Este capítulo serve como base estrutural do projeto.  
A clareza na definição do problema e dos requisitos evita desperdício de tempo,
orienta o planejamento correto e fornece uma visão objetiva do que será
construído nos próximos capítulos.

Um projeto bem-sucedido começa com requisitos bem definidos.
