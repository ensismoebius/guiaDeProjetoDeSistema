\chapter{Arquitetura e Modelagem}
A arquitetura define como o sistema será organizado internamente e como suas
partes irão se comunicar. Já a modelagem transforma ideias abstratas em
representações visuais e técnicas, que orientam o desenvolvimento.  
Para iniciantes, esta etapa é essencial para evitar bagunça, retrabalho e
decisões improvisadas.

Este capítulo propõe uma abordagem clara, acessível e usada em projetos
profissionais, mas adaptada para estudantes no início da jornada.

%%%%%%%%%%%%%%%%%%%%%%%%%%%%%%%%%%%%%%%%%%%%%%%%%%%%%%%%%%%%%%%%%%%%
\section*{1. O Que é Arquitetura de Software?}

Arquitetura é a estrutura fundamental do sistema: os blocos principais,
como eles se conectam e como dados circulam entre eles.  
Sua função é garantir clareza, organização e uma base para o crescimento
do software.

Benefícios:

\begin{itemize}
	\item facilita manutenção e evolução;
	\item evita duplicação de funcionalidades;
	\item ajuda a encontrar erros mais cedo;
	\item torna o projeto compreensível para todos.
\end{itemize}

Sem arquitetura, iniciantes rapidamente criam “código espalhado”, difícil de
testar e impossível de melhorar.

%%%%%%%%%%%%%%%%%%%%%%%%%%%%%%%%%%%%%%%%%%%%%%%%%%%%%%%%%%%%%%%%%%%%
\section*{2. O Que é Modelagem de Software?}

Modelagem é o processo de criar representações visuais ou textuais do sistema.
Ela ajuda os alunos a "ver" o que estão construindo antes do código.

Principais modelos:

\begin{itemize}
	\item Diagrama de Telas.
	\item Diagrama de Navegação.
	\item Diagrama Entidade-Relacionamento (ER).
	\item Diagrama de Componentes.
	\item Diagrama de Fluxo (Flowchart).
\end{itemize}

Esses modelos funcionam como um “mapa” de todo o sistema.

%%%%%%%%%%%%%%%%%%%%%%%%%%%%%%%%%%%%%%%%%%%%%%%%%%%%%%%%%%%%%%%%%%%%
\section*{3. Como Definir uma Arquitetura Simples para Iniciantes}

A arquitetura mais adequada para alunos iniciantes é a separação em três camadas:

\begin{itemize}
	\item \textbf{\gls{frontend}}: interface com o usuário.
	\item \textbf{\gls{backend}}: regras de negócio e lógica da aplicação.
	\item \textbf{Banco de Dados}: armazenamento de informações.
\end{itemize}

Esse modelo é simples, amplamente usado e funciona em 99\% dos projetos de aula.

%%%%%%%%%%%%%%%%%%%%%%%%%%%%%%%%%%%%%%%%%%%%%%%%%%%%%%%%%%%%%%%%%%%%
\section*{4. Fluxo de Dados Entre as Camadas}

Um fluxo típico:

\begin{enumerate}
	\item Usuário clica em um botão na interface.
	\item O \gls{frontend} envia uma requisição ao \gls{backend}.
	\item O \gls{backend} processa a lógica e valida informações.
	\item O \gls{backend} acessa o banco de dados quando necessário.
	\item O \gls{backend} devolve uma resposta ao \gls{frontend}.
	\item O \gls{frontend} atualiza a tela com o resultado.
\end{enumerate}

Com esse fluxo claro, cada aluno sabe onde seu trabalho se encaixa.

%%%%%%%%%%%%%%%%%%%%%%%%%%%%%%%%%%%%%%%%%%%%%%%%%%%%%%%%%%%%%%%%%%%%
\section*{5. Escolha de Tecnologias Simples}

Por se tratar de alunos iniciantes, recomenda-se trabalhar com tecnologias:

\begin{itemize}
	\item intuitivas;
	\item com boa documentação;
	\item com grande comunidade;
	\item fáceis de testar;
	\item fáceis de instalar.
\end{itemize}

Exemplos usuais em projetos didáticos: HTML/CSS/JS, Python, JavaScript no \gls{backend}, SQLite ou MariaDB.

O objetivo não é complexidade técnica, mas aprendizagem clara.

%%%%%%%%%%%%%%%%%%%%%%%%%%%%%%%%%%%%%%%%%%%%%%%%%%%%%%%%%%%%%%%%%%%%
\section*{6. Modelagem de Dados (Entidades e Atributos)}

Antes de programar, defina as entidades principais.  
Cada entidade deve representar um conceito do mundo real, como:

\begin{itemize}
	\item Cliente
	\item Produto
	\item Usuário
	\item Pedido
	\item Item
\end{itemize}

Depois, defina os atributos de cada entidade.  
Exemplo para “Usuário”:

\begin{itemize}
	\item id\_usuario (chave primária)
	\item nome
	\item email
	\item senha
	\item data\_cadastro
\end{itemize}

%%%%%%%%%%%%%%%%%%%%%%%%%%%%%%%%%%%%%%%%%%%%%%%%%%%%%%%%%%%%%%%%%%%%
\section*{7. Diagramas Recomendados}

Para iniciantes, dois diagramas são mais importantes:

\subsection*{7.1 Diagrama de Telas (Wireframe)}
Mostra como cada tela será organizada.

\subsection*{7.2 Diagrama ER}
Mostra as entidades e seus relacionamentos, por exemplo:

\begin{itemize}
	\item Um cliente pode ter muitos pedidos.
	\item Um pedido tem vários itens.
\end{itemize}

%%%%%%%%%%%%%%%%%%%%%%%%%%%%%%%%%%%%%%%%%%%%%%%%%%%%%%%%%%%%%%%%%%%%
\section*{8. Exemplo de Diagrama em TikZ}

A seguir um exemplo compacto de arquitetura:

\begin{tikzpicture}[node distance=2cm, auto]
	\node[rectangle, draw] (frontend) {Frontend};
	\node[rectangle, draw, right=of frontend] (api) {Backend / API};
	\node[rectangle, draw, right=of api] (db) {Banco de Dados};
	\draw[->] (frontend) -- (api);
	\draw[->] (api) -- (db);
\end{tikzpicture}

%%%%%%%%%%%%%%%%%%%%%%%%%%%%%%%%%%%%%%%%%%%%%%%%%%%%%%%%%%%%%%%%%%%%
\section*{9. Padrões de Arquitetura Simples}

Mesmo iniciantes podem usar pequenos padrões:

\begin{itemize}
	\item divisão em camadas;
	\item separação de responsabilidades;
	\item funções pequenas e específicas;
	\item rotas organizadas;
	\item nomes claros para variáveis e arquivos.
\end{itemize}

Não é necessário conhecer padrões complexos como MVC ou microserviços, mas a
disciplina de organizar o código desde o início faz enorme diferença.

%%%%%%%%%%%%%%%%%%%%%%%%%%%%%%%%%%%%%%%%%%%%%%%%%%%%%%%%%%%%%%%%%%%%
\section*{10. Riscos Arquiteturais Comuns para Iniciantes}

\begin{itemize}
	\item colocar toda a lógica em um único arquivo;
	\item misturar HTML com lógica sem organização;
	\item criar \ac{api}s sem padronização de rotas;
	\item usar nomes genéricos como “teste1.js”, “codigoFinal2.py”.
\end{itemize}

Prevenir esses erros é papel da arquitetura.

%%%%%%%%%%%%%%%%%%%%%%%%%%%%%%%%%%%%%%%%%%%%%%%%%%%%%%%%%%%%%%%%%%%%
\section*{11. Checklist do Capítulo}

\begin{itemize}
	\item Todos os alunos entendem a arquitetura geral?
	\item O sistema foi dividido em camadas?
	\item Existe ao menos um diagrama de telas?
	\item Existe um diagrama de entidades?
	\item As tecnologias foram definidas?
	\item Os relacionamentos entre componentes estão claros?
\end{itemize}

%%%%%%%%%%%%%%%%%%%%%%%%%%%%%%%%%%%%%%%%%%%%%%%%%%%%%%%%%%%%%%%%%%%%
\section*{12. Instruções Passo-a-Passo (Para Leigos)}

\begin{enumerate}
	\item Pegue papel ou abra um arquivo digital.
	\item Desenhe as telas principais do sistema.
	\item Liste as entidades e os dados que cada uma deve guardar.
	\item Conecte as entidades com linhas (relacionamentos).
	\item Escolha uma tecnologia simples para cada camada.
	\item Desenhe o fluxo: \gls{frontend} → \gls{backend} → banco.
	\item Revise com o professor antes de iniciar o código.
\end{enumerate}

%%%%%%%%%%%%%%%%%%%%%%%%%%%%%%%%%%%%%%%%%%%%%%%%%%%%%%%%%%%%%%%%%%%%
\section*{13. Encerramento do Capítulo}

A arquitetura orienta todo o projeto, e a modelagem garante que todos tenham a
mesma visão do sistema. Quando os alunos têm um modelo visual do que será
construído, o desenvolvimento flui melhor, os erros diminuem e o aprendizado se
torna mais concreto.
