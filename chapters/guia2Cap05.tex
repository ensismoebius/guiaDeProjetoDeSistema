\chapter{Desenvolvimento do \ac{mvp}}
O \ac{mvp} (Produto Mínimo Viável) é a primeira versão funcional do sistema, contendo
apenas o essencial para demonstrar que as funcionalidades básicas existem e
funcionam.  
Esta é a etapa em que o sistema “ganha vida”.  
Para alunos iniciantes, desenvolver um \ac{mvp} bem estruturado é a melhor forma de
aprender a transformar requisitos em código real.

%%%%%%%%%%%%%%%%%%%%%%%%%%%%%%%%%%%%%%%%%%%%%%%%%%%%%%%%%%%%%%%%%%%%
\section*{1. O Que é o \ac{mvp} e Por Que Ele Importa?}

O \ac{mvp} é a menor versão utilizável do sistema.  
Ele contém somente funcionalidades centrais, sem detalhes, sem estética avançada
e sem otimizações.  

Benefícios para iniciantes:

\begin{itemize}
	\item reduz ansiedade por não exigir perfeição;
	\item cria sensação de progresso rápido;
	\item diminui retrabalho;
	\item permite testar conceitos antes do sistema completo;
	\item estabelece um ponto firme para expansões posteriores.
\end{itemize}

Focar no \ac{mvp} evita que alunos gastem semanas em detalhes antes de ter algo funcionando.

%%%%%%%%%%%%%%%%%%%%%%%%%%%%%%%%%%%%%%%%%%%%%%%%%%%%%%%%%%%%%%%%%%%%
\section*{2. Seleção das Funcionalidades Essenciais}

Para definir o que entra no \ac{mvp}:

\begin{enumerate}
	\item Liste todas as funcionalidades do sistema.
	\item Marque quais são essenciais para o funcionamento básico.
	\item Remova tudo que for opcional, decorativo ou avançado.
	\item Garanta que o \ac{mvp} possa ser demonstrado em 1–2 minutos.
\end{enumerate}

Exemplos de funcionalidades essenciais:

\begin{itemize}
	\item autenticação simples (login);
	\item cadastro básico de entidades principais;
	\item listagem de itens;
	\item operação mínima de criação/edição;
	\item fluxo funcional de ponta a ponta.
\end{itemize}

%%%%%%%%%%%%%%%%%%%%%%%%%%%%%%%%%%%%%%%%%%%%%%%%%%%%%%%%%%%%%%%%%%%%
\section*{3. Estrutura Modular para o \ac{mvp}}

Mesmo o \ac{mvp} deve ser dividido em módulos pequenos e independentes, como:

\begin{itemize}
	\item módulo de cadastro;
	\item módulo de listagem;
	\item módulo de autenticação;
	\item módulo de interface inicial;
	\item módulo de comunicação com o banco.
\end{itemize}

Para alunos iniciantes, modularidade:

\begin{itemize}
	\item facilita testes;
	\item permite que grupos trabalhem em paralelo;
	\item reduz erros;
	\item aumenta clareza.
\end{itemize}

%%%%%%%%%%%%%%%%%%%%%%%%%%%%%%%%%%%%%%%%%%%%%%%%%%%%%%%%%%%%%%%%%%%%
\section*{4. Desenvolvimento Iterativo e Incremental}

O \ac{mvp} deve ser construído passo a passo:

\begin{enumerate}
	\item Criar rota simples.
	\item Fazer \gls{endpoint} responder algo básico.
	\item Exibir esse resultado no \gls{frontend}.
	\item Salvar algo simples no banco.
	\item Validar fluxo completo.
\end{enumerate}

Cada etapa deve funcionar antes de avançar para a próxima.

%%%%%%%%%%%%%%%%%%%%%%%%%%%%%%%%%%%%%%%%%%%%%%%%%%%%%%%%%%%%%%%%%%%%
\section*{5. Primeiro Ciclo de Implementação}

Sugestão prática para iniciar o \ac{mvp}:

\subsection*{5.1 \gls{backend}}
\begin{itemize}
	\item Criar rota “/status”.
	\item Implementar retorno fixo: \texttt{\{"ok": true\}}.
	\item Criar rota de listagem de uma entidade.
	\item Retornar dados estáticos.
\end{itemize}

\subsection*{5.2 \gls{frontend}}
\begin{itemize}
	\item Criar botão “Testar Conexão”.
	\item Exibir resposta da rota \texttt{/status}.
	\item Criar tela simples de listagem.
	\item Inserir dados fictícios.
\end{itemize}

\subsection*{5.3 Banco de Dados}
\begin{itemize}
	\item Criar tabela mínima.
	\item Inserir 1 ou 2 registros de teste.
	\item Ligar rota do \gls{backend} ao banco.
\end{itemize}

%%%%%%%%%%%%%%%%%%%%%%%%%%%%%%%%%%%%%%%%%%%%%%%%%%%%%%%%%%%%%%%%%%%%
\section*{6. Segundo Ciclo de Implementação}

Após o primeiro fluxo básico:

\begin{itemize}
	\item substituir dados falsos por consultas reais;
	\item implementar inserção básica;
	\item implementar edição básica;
	\item validar regras simples (ex.: campos obrigatórios);
	\item controlar pequenos erros.
\end{itemize}

Essas evoluções devem ser feitas gradualmente.

%%%%%%%%%%%%%%%%%%%%%%%%%%%%%%%%%%%%%%%%%%%%%%%%%%%%%%%%%%%%%%%%%%%%
\section*{7. Boas Práticas para Iniciantes}

\begin{itemize}
	\item Comece sempre pelo mais simples.
	\item Use nomes claros para arquivos e funções.
	\item Escreva código pequeno, funções curtas.
	\item Teste cada pequena parte antes de avançar.
	\item Evite copiar e colar código sem entender.
	\item Salve progresso com commits frequentes.
\end{itemize}

%%%%%%%%%%%%%%%%%%%%%%%%%%%%%%%%%%%%%%%%%%%%%%%%%%%%%%%%%%%%%%%%%%%%
\section*{8. Riscos Comuns no Desenvolvimento do \ac{mvp}}

\begin{itemize}
	\item tentar fazer funcionalidades avançadas cedo demais;
	\item criar telas complexas antes de ter lógica funcionando;
	\item não testar o fluxo completo;
	\item deixar o banco mal definido;
	\item acúmulo de código improvisado.
\end{itemize}

%%%%%%%%%%%%%%%%%%%%%%%%%%%%%%%%%%%%%%%%%%%%%%%%%%%%%%%%%%%%%%%%%%%%
\section*{9. Checklist do Capítulo}

Ao terminar este capítulo, os alunos devem:

\begin{itemize}
	\item ter um \ac{mvp} funcional;
	\item mostrar pelo menos um fluxo completo;
	\item ter módulos mínimos implementados;
	\item conseguir demonstrar o sistema em 1–2 minutos;
	\item saber quais partes serão expandidas depois.
\end{itemize}

%%%%%%%%%%%%%%%%%%%%%%%%%%%%%%%%%%%%%%%%%%%%%%%%%%%%%%%%%%%%%%%%%%%%
\section*{10. Instruções Passo-a-Passo (Para Leigos)}

\begin{enumerate}
	\item Liste apenas as funções essenciais.
	\item Desenhe uma tela simples para cada função.
	\item Crie uma rota de teste no \gls{backend}.
	\item Faça o \gls{frontend} chamar essa rota.
	\item Crie uma tabela simples no banco.
	\item Conecte uma rota real ao banco.
	\item Teste tudo clicando na interface.
	\item Repita o ciclo sempre melhorando aos poucos.
\end{enumerate}

%%%%%%%%%%%%%%%%%%%%%%%%%%%%%%%%%%%%%%%%%%%%%%%%%%%%%%%%%%%%%%%%%%%%
\section*{11. Encerramento do Capítulo}

O \ac{mvp} marca o momento em que o projeto deixa de ser apenas planejamento e passa
a ser um sistema real, visível e utilizável.  
Para iniciantes, ele representa um divisor de águas: mostra que é possível
construir software, reduz medos e fortalece o aprendizado prático.
