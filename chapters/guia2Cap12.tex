\chapter{Desenvolvimento Seguro e Boas Práticas de Proteção}

Este capítulo apresenta práticas fundamentais de segurança que podem (e devem)
ser aplicadas mesmo por equipes iniciantes.  
Ao contrário do que muitos alunos pensam, segurança não é um tema avançado:
é um conjunto de hábitos básicos que evitam falhas graves no código e no uso
do sistema.

%%%%%%%%%%%%%%%%%%%%%%%%%%%%%%%%%%%%%%%%%%%%%%%%%%%%%%%%%%%%%%%%%%%%
\section*{1. Por Que Segurança Importa Mesmo em Projetos Escolares?}

Falhas simples podem causar:

\begin{itemize}
	\item perda de dados;
	\item acesso não autorizado;
	\item travamentos e comportamento inesperado;
	\item danos à apresentação final;
	\item impossibilidade de demonstrar o sistema funcionando.
\end{itemize}

Mesmo um pequeno projeto precisa:

\begin{itemize}
	\item validar tudo que o usuário digita;
	\item controlar quem tem acesso a cada parte do sistema;
	\item impedir que ações indevidas se tornem possíveis.
\end{itemize}

%%%%%%%%%%%%%%%%%%%%%%%%%%%%%%%%%%%%%%%%%%%%%%%%%%%%%%%%%%%%%%%%%%%%
\section*{2. Princípios Fundamentais de Segurança para Iniciantes}

\subsection*{2.1 Validação de Entrada (Input Validation)}
Todo dado vindo do usuário é duvidoso até ser validado.

Valide sempre:

\begin{itemize}
	\item campos de texto;
	\item números;
	\item seleções;
	\item uploads;
	\item dados vindos da rede.
\end{itemize}

Exemplos de erros prevenidos:

\begin{itemize}
	\item dados fora do formato esperado;
	\item SQL com caracteres inválidos;
	\item travamentos por strings maiores que o permitido.
\end{itemize}

\subsection*{2.2 Princípio do Menor Privilégio}
Cada parte do sistema deve ter permissões apenas para o que realmente precisa.

\begin{itemize}
	\item usuários comuns não podem acessar páginas administrativas;
	\item funções simples não devem ter acesso ao banco inteiro;
	\item arquivos sensíveis não devem ser públicos.
\end{itemize}

\subsection*{2.3 Proteção de Dados Sensíveis}
Dados como senhas, identificadores e informações de clientes precisam ser:

\begin{itemize}
	\item armazenados de forma segura;
	\item nunca impressos no console;
	\item nunca enviados em texto puro.
\end{itemize}

%%%%%%%%%%%%%%%%%%%%%%%%%%%%%%%%%%%%%%%%%%%%%%%%%%%%%%%%%%%%%%%%%%%%
\section*{3. Erros de Segurança Comuns em Projetos de Alunos}

Comportamentos frequentes que devem ser evitados:

\begin{itemize}
	\item campos sem validação ("deixa assim mesmo");
	\item mensagens de erro que revelam detalhes técnicos;
	\item URLs sensíveis sem autenticação;
	\item senhas fracas hardcoded em arquivos;
	\item falta de verificação de permissão ao mudar de tela;
	\item envio de dados sem criptografia;
	\item campos numéricos aceitando letras.
\end{itemize}

Cada um desses pontos pode comprometer o sistema completamente.

%%%%%%%%%%%%%%%%%%%%%%%%%%%%%%%%%%%%%%%%%%%%%%%%%%%%%%%%%%%%%%%%%%%%
\section*{4. Como Validar Entrada de Forma Simples e Segura}

Para cada campo:

\begin{itemize}
	\item verifique se está preenchido;
	\item confirme o tipo (texto, número, data);
	\item defina tamanhos mínimos e máximos;
	\item aplique expressões regulares quando necessário;
	\item mostre mensagens claras ao usuário.
\end{itemize}

Exemplo de validação manual:

\begin{verbatim}
	Se telefone tiver menos que 8 dígitos → erro.
	Se CPF tiver letras → erro.
	Se nome tiver números → erro.
\end{verbatim}

%%%%%%%%%%%%%%%%%%%%%%%%%%%%%%%%%%%%%%%%%%%%%%%%%%%%%%%%%%%%%%%%%%%%
\section*{5. Autenticação e Controle de Acesso em Projetos Simples}

Mesmo projetos pequenos precisam de:

\begin{itemize}
	\item uma tela de login;
	\item funções restritas para usuários autenticados;
	\item níveis diferentes de permissão quando necessário;
	\item registro automático de tentativas inválidas.
\end{itemize}

Regras básicas:

\begin{itemize}
	\item senhas nunca devem ser exibidas;
	\item após 3 tentativas erradas, exibir mensagem educada;
	\item logout deve limpar toda sessão do usuário.
\end{itemize}

%%%%%%%%%%%%%%%%%%%%%%%%%%%%%%%%%%%%%%%%%%%%%%%%%%%%%%%%%%%%%%%%%%%%
\section*{6. Proteção Contra Operações Críticas}

Qualquer ação que altere dados deve solicitar confirmação, como:

\begin{itemize}
	\item deletar registros;
	\item editar dados importantes;
	\item cancelar contratos;
	\item resetar configurações;
	\item enviar informações sensíveis.
\end{itemize}

Exemplo de diálogo:

\begin{verbatim}
	"Tem certeza que deseja excluir este cliente?"
	[Sim] [Não]
\end{verbatim}

%%%%%%%%%%%%%%%%%%%%%%%%%%%%%%%%%%%%%%%%%%%%%%%%%%%%%%%%%%%%%%%%%%%%
\section*{7. Registro e Monitoramento}

Mesmo sistemas pequenos devem registrar:

\begin{itemize}
	\item logins bem-sucedidos;
	\item tentativas de login incorretas;
	\item erros de validação;
	\item operações críticas (deleções, edições, inserções);
	\item cenários inesperados.
\end{itemize}

Isso ajuda a identificar:

\begin{itemize}
	\item comportamentos estranhos;
	\item falhas repetitivas;
	\item alunos que precisam de ajuda;
	\item funcionalidades usadas com frequência.
\end{itemize}

%%%%%%%%%%%%%%%%%%%%%%%%%%%%%%%%%%%%%%%%%%%%%%%%%%%%%%%%%%%%%%%%%%%%
\section*{8. Segurança Durante o Desenvolvimento}

Práticas essenciais:

\begin{itemize}
	\item não usar senhas reais em ambientes de teste;
	\item não publicar chaves em repositórios públicos;
	\item não deixar prints de depuração em produção;
	\item não usar dados confidenciais como exemplo.
\end{itemize}

%%%%%%%%%%%%%%%%%%%%%%%%%%%%%%%%%%%%%%%%%%%%%%%%%%%%%%%%%%%%%%%%%%%%
\section*{9. Checklist de Segurança do Capítulo}

\begin{itemize}
	\item todas as entradas têm validação?
	\item existe controle de acesso por tela?
	\item operações críticas têm confirmação?
	\item senhas são mantidas em segurança?
	\item existe registro de eventos importantes?
	\item erros exibidos ao usuário não revelam detalhes internos?
	\item o sistema foi testado por outro colega em busca de falhas?
\end{itemize}

%%%%%%%%%%%%%%%%%%%%%%%%%%%%%%%%%%%%%%%%%%%%%%%%%%%%%%%%%%%%%%%%%%%%
\section*{10. Instruções Passo-a-Passo (Para Leigos)}

\begin{enumerate}
	\item Abra cada tela da aplicação.
	\item Teste campos enviando dados errados de propósito.
	\item Veja se o sistema detecta o erro.
	\item Tente acessar telas avançadas sem estar logado.
	\item Tente excluir algo sem confirmação.
	\item Verifique se mensagens de erro são claras.
	\item Peça para um colega tentar “quebrar” o sistema.
	\item Anote tudo que parecer inseguro.
	\item Corrija o que for possível.
\end{enumerate}

%%%%%%%%%%%%%%%%%%%%%%%%%%%%%%%%%%%%%%%%%%%%%%%%%%%%%%%%%%%%%%%%%%%%
\section*{11. Encerramento do Capítulo}

Segurança não é uma camada extra: é parte natural do desenvolvimento.
Aplicar boas práticas desde o início evita falhas graves, aumenta a
qualidade do software e reduz retrabalho.  
Alunos que aprendem segurança cedo constroem sistemas mais robustos,
responsáveis e confiáveis.
