\chapter{Preparação Inicial}
Este capítulo estabelece a base técnica e organizacional do projeto. 
Seu objetivo é preparar cada aluno iniciante para trabalhar de forma 
coordenada, com ferramentas funcionando, regras claras e segurança 
no uso dos primeiros recursos de desenvolvimento.

%%%%%%%%%%%%%%%%%%%%%%%%%%%%%%%%%%%%%%%%%%%%%%%%%%%%%%%%%%%%%%%%%%%%
\section*{1. Objetivo da Preparação Inicial}

A preparação inicial impede que problemas básicos atrapalhem as etapas
mais importantes do projeto. Quando a turma inicia com um ambiente unificado
e práticas bem definidas, o risco de retrabalho, confusão e atrasos reduz
drasticamente.

Os objetivos principais desta fase são:

\begin{itemize}
	\item garantir que todos os alunos tenham o mesmo ambiente configurado;
	\item apresentar as ferramentas essenciais do curso;
	\item padronizar desde o início a forma de trabalhar;
	\item ensinar o fluxo de contribuição com clareza;
	\item distribuir papéis que promovam organização e responsabilidade.
\end{itemize}

Esta padronização inicial é um dos fatores mais fortes de sucesso em projetos
colaborativos grandes e pequenos.

%%%%%%%%%%%%%%%%%%%%%%%%%%%%%%%%%%%%%%%%%%%%%%%%%%%%%%%%%%%%%%%%%%%%
\section*{2. Ambiente Técnico}

Ambientes complicados criam barreiras desnecessárias para quem está começando.
Por isso, o ambiente técnico sugerido é simples, robusto e amplamente testado.

Ferramentas recomendadas:

\begin{itemize}
	\item \textbf{Editor}: VS Code (pela facilidade de configuração);
	\item \textbf{Controle de versão}: \ac{git}, integrado ao GitHub;
	\item \textbf{Terminal}: o nativo do sistema operacional;
	\item \textbf{Navegador}: Firefox ou Chromium.
\end{itemize}

Extensões essenciais no VS Code:

\begin{itemize}
	\item GitLens (visualização de histórico);
	\item Linter e formatador da linguagem utilizada;
	\item Extensão de depuração;
	\item Extensão para trabalho com Markdown.
\end{itemize}

Com esse conjunto mínimo, todos conseguem instalar, abrir, editar e versionar
arquivos sem atrito.

%%%%%%%%%%%%%%%%%%%%%%%%%%%%%%%%%%%%%%%%%%%%%%%%%%%%%%%%%%%%%%%%%%%%
\section*{3. Organização e Padronização}

Equipes iniciantes avançam mais rápido quando seguem regras explícitas. Nada é
óbvio para quem está começando. Por isso, a padronização deve ser clara,
objetiva e obrigatória.

\textbf{Padrões que devem ser fixados desde o primeiro dia:}

\begin{itemize}
	\item formato de commits;
	\item convenção de nomes de \gls{branch};
	\item estrutura de pastas do projeto;
	\item modelo de \glspl{issue};
	\item modelo de Pull Requests;
	\item estilo de código (indentação, nomes de arquivos, etc.).
\end{itemize}

\textbf{Documentação mínima obrigatória:}  
Pequena, mas constante. Cada alteração relevante deve gerar alguma forma de
documentação básica: uma \gls{issue} fechada, um comentário de \ac{pr}, um trecho de README.

\textbf{Canal único de comunicação:}  
A turma deve usar um único canal oficial (como Telegram/Discord) com regras bem
definidas. Isso evita fragmentação de informações.

%%%%%%%%%%%%%%%%%%%%%%%%%%%%%%%%%%%%%%%%%%%%%%%%%%%%%%%%%%%%%%%%%%%%
\section*{4. Fluxo de Trabalho Inicial}

O fluxo inicial deve ser simples e repetível, para que todos memorizem sem
dificuldade. A seguir está o fluxo recomendado:

\begin{enumerate}
	\item O professor cria o repositório principal.
	\item Cada aluno recebe acesso ou faz um fork.
	\item O aluno clona o repositório para sua máquina.
	\item Cria uma \gls{branch} para testes.
	\item Faz um \gls{commit} simples para aprender o fluxo.
	\item Envia (push) para o GitHub.
	\item Abre uma Pull Request.
	\item Recebe comentários e ajusta o código.
\end{enumerate}

Esse processo, repetido no início, consolida o uso básico do \ac{git} e reduz
bloqueios futuros.

%%%%%%%%%%%%%%%%%%%%%%%%%%%%%%%%%%%%%%%%%%%%%%%%%%%%%%%%%%%%%%%%%%%%
\section*{5. Distribuição de Papéis da Turma}

Mesmo em equipes iniciantes, dividir papéis reduz desorganização e aumenta
responsabilidade individual. Cada grupo pode assumir um conjunto de papéis,
trocar após duas semanas e assim adquirir experiência prática diversificada.

Papéis sugeridos:

\begin{itemize}
	\item \textbf{Coordenador de Sprint} — organiza tarefas e acompanha progresso;
	\item \textbf{Gestor de \glspl{issue}} — cria, mantém e valida clareza das tarefas;
	\item \textbf{Responsável por Integração} — auxilia merges e resolução de conflitos;
	\item \textbf{Responsável por Testes} — supervisiona a criação e execução de testes;
	\item \textbf{Documentador} — mantém documentação atualizada.
\end{itemize}

Esta rotação gradativa cria um ambiente onde todos aprendem todas as funções,
evitando dependência excessiva em um único aluno.

%%%%%%%%%%%%%%%%%%%%%%%%%%%%%%%%%%%%%%%%%%%%%%%%%%%%%%%%%%%%%%%%%%%%
\section*{6. Problemas Comuns (E Como Evitá-los)}

Projetos iniciantes apresentam um conjunto de dificuldades previsíveis:

\textbf{Problema:} ambiente mal configurado.  
\textbf{Solução:} guia com capturas de tela e verificação inicial em duplas.

\textbf{Problema:} erros constantes com \ac{git}.  
\textbf{Solução:} ensinar apenas 5 comandos no início:
\texttt{git status}, \texttt{git add}, \texttt{git commit}, \texttt{git push}, \texttt{git pull}.

\textbf{Problema:} branches confusas.  
\textbf{Solução:} uso obrigatório das convenções:  
\begin{itemize}
    \item \texttt{feature/nome}
    \item \texttt{bugfix/nome}
    \item \texttt{hotfix/nome}
\end{itemize}

\textbf{Problema:} PRs grandes e difíceis de revisar.  
\textbf{Solução:} limitar PRs a 200 linhas, sempre que possível.

%%%%%%%%%%%%%%%%%%%%%%%%%%%%%%%%%%%%%%%%%%%%%%%%%%%%%%%%%%%%%%%%%%%%
\section*{7. Instruções Passo-a-Passo para Leigos}

A seguir, uma versão ampliada do passo-a-passo, pensada para quem nunca
programou:

\begin{enumerate}
	\item Instale o VS Code seguindo um tutorial com imagens.
	\item Instale o \ac{git} e teste rodando \texttt{git --version}.
	\item Crie uma conta no GitHub e faça login.
	\item Abra o repositório fornecido e clique em “Fork” ou aceite o acesso.
	\item Copie o comando “git clone” que aparece na página e execute no terminal.
	\item Abra a pasta clonada no VS Code.
	\item Crie um arquivo simples, como \texttt{teste.txt}, e escreva duas linhas.
	\item No VS Code, clique em “Source Control” e faça o \gls{commit}.
	\item Clique em “Sync” para enviar as mudanças.
	\item Abra uma Pull Request explicando: “Criado arquivo de teste”.
	\item Aguarde comentários.
	\item Ajuste o arquivo e envie novamente.
	\item Pronto: você fez sua primeira contribuição real!
\end{enumerate}

%%%%%%%%%%%%%%%%%%%%%%%%%%%%%%%%%%%%%%%%%%%%%%%%%%%%%%%%%%%%%%%%%%%%
\section*{8. Encerramento do Capítulo}

Ao final desta etapa, cada aluno deve:

\begin{itemize}
	\item ter o ambiente configurado;
	\item saber criar branches e commits;
	\item entender o fluxo básico de contribuição;
	\item conhecer os papéis do time;
	\item estar pronto para começar a etapa de requisitos.
\end{itemize}

Este capítulo serve como base para todo o projeto.
