\chapter{Testes, Qualidade e Garantia de Funcionamento}

Garantir qualidade é um dos pilares essenciais em qualquer projeto de software.
Mesmo quando desenvolvido por alunos iniciantes, o sistema deve funcionar de
forma confiável, previsível e estável.  
Este capítulo apresenta práticas simples, diretas e eficazes de testes e
controle de qualidade que podem ser aplicadas sem conhecimento avançado,
produzindo resultados sólidos em pouco tempo.

%%%%%%%%%%%%%%%%%%%%%%%%%%%%%%%%%%%%%%%%%%%%%%%%%%%%%%%%%%%%%%%%%%%%
\section*{1. O Que São Testes de Software?}

Testar software significa verificar se ele funciona como deveria.

Objetivos dos testes:

\begin{itemize}
	\item encontrar erros cedo;
	\item garantir que cada funcionalidade faz o que foi planejado;
	\item evitar que novos recursos quebrem partes antigas;
	\item melhorar a confiança no sistema;
	\item reduzir retrabalho no final do projeto.
\end{itemize}

Para iniciantes, testar é parte natural do processo de aprender a programar.

%%%%%%%%%%%%%%%%%%%%%%%%%%%%%%%%%%%%%%%%%%%%%%%%%%%%%%%%%%%%%%%%%%%%
\section*{2. Tipos Básicos de Teste}

Os alunos devem praticar ao menos os seguintes tipos:

\subsection*{2.1 Teste Manual}
O aluno usa a aplicação como se fosse um usuário real e verifica:

\begin{itemize}
	\item fluxo das telas;
	\item validação de campos;
	\item confirmações e alertas;
	\item erros e travamentos;
	\item funcionalidades completas do início ao fim.
\end{itemize}

\subsection*{2.2 Testes de Unidade}
Testam pequenas partes do código (funções ou métodos).
Para iniciantes, o professor pode entregar modelos prontos para completar.

\subsection*{2.3 Testes de Integração}
Checam se módulos diferentes funcionam bem juntos:

\begin{itemize}
	\item tela + \gls{backend};
	\item \gls{backend} + banco de dados;
	\item diferentes partes do sistema acessando a mesma informação.
\end{itemize}

\subsection*{2.4 Testes de Regressão}
Garantem que funcionalidades antigas continuam funcionando após novas mudanças.  
São essenciais ao final do semestre.

%%%%%%%%%%%%%%%%%%%%%%%%%%%%%%%%%%%%%%%%%%%%%%%%%%%%%%%%%%%%%%%%%%%%
\section*{3. Critérios de Qualidade para Iniciantes}

Três critérios são suficientes para projetos do nível da turma:

\subsection*{3.1 Funcionalidade}
Tudo que foi planejado precisa:

\begin{itemize}
	\item existir;
	\item funcionar corretamente;
	\item ser acessível ao usuário.
\end{itemize}

\subsection*{3.2 Confiabilidade}
A aplicação não pode travar, perder dados ou quebrar ao:

\begin{itemize}
	\item digitar dados incorretos;
	\item clicar rápido demais;
	\item repetir ações;
	\item usar em computadores diferentes.
\end{itemize}

\subsection*{3.3 Usabilidade}
O sistema deve ser simples de usar:

\begin{itemize}
	\item botões claros;
	\item fluxos intuitivos;
	\item mensagens compreensíveis;
	\item sem termos técnicos desnecessários.
\end{itemize}

%%%%%%%%%%%%%%%%%%%%%%%%%%%%%%%%%%%%%%%%%%%%%%%%%%%%%%%%%%%%%%%%%%%%
\section*{4. Como Criar um Plano de Testes Simples}

Cada grupo deve criar um documento com:

\begin{itemize}
	\item funcionalidades a serem testadas;
	\item passos detalhados para executar cada teste;
	\item resultado esperado;
	\item resultado obtido;
	\item prints da tela;
	\item status (aprovado/reprovado).
\end{itemize}

Exemplo:

\begin{verbatim}
	Teste: cadastrar cliente
	Passos:
	1. Abrir tela de cadastro.
	2. Preencher nome, CPF e telefone.
	3. Clicar "Salvar".
	
	Esperado: cliente aparece na lista.
	Obtido: cliente salvo corretamente.
	Status: aprovado.
\end{verbatim}

%%%%%%%%%%%%%%%%%%%%%%%%%%%%%%%%%%%%%%%%%%%%%%%%%%%%%%%%%%%%%%%%%%%%
\section*{5. Registro de \glspl{bug} (\gls{bug} Report)}

Quando o aluno encontra um erro, deve registrar uma \gls{issue} contendo:

\begin{itemize}
	\item descrição clara do problema;
	\item passos para reproduzir;
	\item comportamento esperado;
	\item comportamento observado;
	\item prints ou erros do console;
	\item \gls{commit} onde o erro foi identificado.
\end{itemize}

Esse modelo evita perda de informação e facilita correção.

%%%%%%%%%%%%%%%%%%%%%%%%%%%%%%%%%%%%%%%%%%%%%%%%%%%%%%%%%%%%%%%%%%%%
\section*{6. Como Priorizar \glspl{bug}}

Nem todos os erros são iguais.

\subsection*{6.1 Erros Críticos}
Travamentos, perda de dados, impossibilidade de usar a aplicação.

\subsection*{6.2 Erros Moderados}
Funcionalidades importantes com comportamento incorreto.

\subsection*{6.3 Erros Estéticos}
Problemas de layout, textos incorretos, pequenos desalinhamentos.

A regra geral:

\begin{itemize}
	\item corrigir primeiro os críticos;
	\item depois os moderados;
	\item por último os estéticos.
\end{itemize}

%%%%%%%%%%%%%%%%%%%%%%%%%%%%%%%%%%%%%%%%%%%%%%%%%%%%%%%%%%%%%%%%%%%%
\section*{7. Ferramentas de Teste Acessíveis a Iniciantes}

Recomenda-se:

\begin{itemize}
	\item Google Forms para registrar testes manuais da equipe;
	\item Checklists em Markdown no GitHub;
	\item Testes automatizados simples com frameworks básicos;
	\item Prints numerados no relatório final.
\end{itemize}

%%%%%%%%%%%%%%%%%%%%%%%%%%%%%%%%%%%%%%%%%%%%%%%%%%%%%%%%%%%%%%%%%%%%
\section*{8. Estratégias para Garantia de Funcionamento}

Para manter o sistema estável até a entrega:

\begin{itemize}
	\item testar toda alteração antes de enviar \gls{commit};
	\item revisar código em dupla;
	\item usar testes de regressão no final de cada sprint;
	\item executar sempre a aplicação do início ao fim semanalmente;
	\item registrar e corrigir \glspl{bug} imediatamente.
\end{itemize}

%%%%%%%%%%%%%%%%%%%%%%%%%%%%%%%%%%%%%%%%%%%%%%%%%%%%%%%%%%%%%%%%%%%%
\section*{9. Testes Realizados por Usuários Reais}

Antes da entrega final, cada grupo deve:

\begin{itemize}
	\item chamar colegas de outra dupla para testar;
	\item observar como a pessoa usa o sistema;
	\item não interferir durante o uso;
	\item registrar dificuldades e confusões;
	\item corrigir antes da apresentação final.
\end{itemize}

Testar com outros alunos revela problemas que o próprio grupo não vê.

%%%%%%%%%%%%%%%%%%%%%%%%%%%%%%%%%%%%%%%%%%%%%%%%%%%%%%%%%%%%%%%%%%%%
\section*{10. Checklist do Capítulo}

\begin{itemize}
	\item as funcionalidades principais foram testadas?
	\item existem registros dos testes?
	\item há um documento de \glspl{bug}?
	\item os erros críticos foram resolvidos?
	\item o sistema funciona do início ao fim?
	\item o grupo testou em computadores diferentes?
	\item foram feitos testes com usuários reais?
\end{itemize}

%%%%%%%%%%%%%%%%%%%%%%%%%%%%%%%%%%%%%%%%%%%%%%%%%%%%%%%%%%%%%%%%%%%%
\section*{11. Instruções Passo-a-Passo (Para Leigos)}

\begin{enumerate}
	\item Abra o sistema.
	\item Liste tudo que ele deveria fazer.
	\item Teste cada funcionalidade.
	\item Anote qualquer problema encontrado.
	\item Tire prints de tela.
	\item Marque se o teste passou ou falhou.
	\item Peça para um colega testar também.
	\item Corrija os problemas.
	\item Teste tudo novamente.
\end{enumerate}

%%%%%%%%%%%%%%%%%%%%%%%%%%%%%%%%%%%%%%%%%%%%%%%%%%%%%%%%%%%%%%%%%%%%
\section*{12. Encerramento do Capítulo}

Testes e qualidade transformam o projeto em software confiável.  
Ao aprender a testar, o aluno aprende a programar melhor, a depurar, a observar
detalhes, a corrigir erros com calma e a melhorar continuamente.  
Um projeto bem testado não é apenas mais seguro: é mais profissional, mais
previsível e mais fácil de apresentar.
