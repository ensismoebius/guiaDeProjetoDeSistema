% --------------------------
% DEFINIÇÃO DE ENTRADAS DO GLOSSÁRIO (usando newglossaryentry)
% Texto em português nas descrições
% --------------------------
\newglossaryentry{frontend}{
  name=frontend,
  description={Camada de interface visível ao usuário, responsável pela apresentação e interação (ex.: HTML, CSS, JavaScript)}
}

\newglossaryentry{backend}{
  name=backend,
  description={Camada de lógica e processamento no servidor, responsável por regras de negócio, armazenamento e API}
}

\newglossaryentry{deploy}{
  name=deploy,
  description={Processo de colocar o sistema em produção ou disponibilizá-lo para uso em um ambiente específico}
}

\newglossaryentry{commit}{
  name=commit,
  description={Registro de alteração no repositório de controle de versão contendo um conjunto coerente de modificações}
}

\newglossaryentry{branch}{
  name=branch,
  description={Linha de desenvolvimento no controle de versão, usada para trabalhar de forma isolada em funcionalidades ou correções}
}

\newglossaryentry{endpoint}{
  name=endpoint,
  description={Ponto de acesso da API que expõe uma funcionalidade específica por meio de uma rota/URL}
}

\newglossaryentry{framework}{
  name=framework,
  description={Estrutura de suporte que provê componentes reutilizáveis e convenções para facilitar o desenvolvimento}
}

\newglossaryentry{pipeline}{
  name=pipeline,
  description={Fluxo automatizado de etapas (build, testes, deploy) que garante repetibilidade e qualidade nas entregas}
}

\newglossaryentry{release}{
  name=release,
  description={Versão oficial do software preparada para distribuição ou entrega, geralmente numerada (ex.: v1.0)}
}

\newglossaryentry{testing}{
  name=testing,
  description={Prática de verificar a funcionalidade do sistema por meio de testes manuais ou automatizados}
}

\newglossaryentry{issue}{
  name=issue,
  description={Registro de tarefa, bug ou solicitação de melhoria em ferramentas de rastreamento (ex.: GitHub Issues)}
}

\newglossaryentry{bug}{name=bug,description={Erro ou falha em um software que causa um comportamento inesperado ou incorreto.}}

\newglossaryentry{userstory}{
  name=user story,
  description={Descrição curta e centrada no usuário de um requisito funcional: formato comum "Como <usuário>, quero <ação> para <benefício>"}
}

\newglossaryentry{kanban}{
  name=Kanban,
  description={Método visual para gerenciar fluxo de trabalho por meio de cartões e colunas (ex.: To Do, Doing, Done)}
}

\newglossaryentry{adrs}{
  name=ADR,
  description={Architectural Decision Record — documentação que registra decisões arquiteturais importantes e suas justificativas}
}

% --------------------------
% Exemplo de entradas adicionais (adicione conforme precisar)
% --------------------------
\newglossaryentry{refactor}{
  name=refactor,
  description={Refatoração: processo de reestruturar código sem alterar seu comportamento observável, para melhorar qualidade}
}

\newglossaryentry{ci-tooling}{
  name={CI tooling},
  description={Ferramentas e serviços que implementam integração contínua (ex.: GitHub Actions, GitLab CI, Jenkins)}
}
