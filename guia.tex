\documentclass[portuguese,12pt,a4paper,final,titlepage]{book}
\usepackage[T1]{fontenc}
\usepackage{graphicx}
\usepackage{babel}
\usepackage{url}
\title{React Native para desenvolvimento mobile}
\author{André Furlan}
\begin{document}
	\maketitle
	
	\chapter{Requisitos de software}
	
		\par Para iniciar o desenvolvimento em React Native é necessário a instalação de alguns programas, abaixo vai uma lista dos mesmos:
		\begin{itemize}
			\item vscode: code.visualstudio.com
			\item extensões do vscode:
			\begin{itemize}
				\item react native tools
				\item expo tools
				\item ES7+ React/Redux/React-Native snippets
				\item vscode-icons (opcional)
			\end{itemize}
			\item git: git-scm.com
			\item nodejs: nodejs.org
			\item expo go: procure por ``expo go'' na \textit{play store}
		\end{itemize}
		
	\chapter{Criando um projeto}
		\par Para iniciar um projeto você pode seguir o passo a passo descrito no endereço: https://docs.expo.dev/get-started/create-a-project/ e, baseado nessa documentação,iremos abordar aqui como fazer.
		
		\section{Criando a base do projeto}
			
			\par Primeiramente, crie um diretório onde \textbf{seus projetos} serão armazenados.
			
			\par Em seguida, abra o terminal e navegue até o diretório recém-criado.  
			
			\par Feito isso, execute o seguinte comando:  
			
			\begin{verbatim}
				npx create-expo-app ddmi
			\end{verbatim}
			
			\par Após executar o comando, será solicitado um nome para o projeto. Escolha um nome sem caracteres especiais, acentos ou espaços.  
			
			\par Ao final do processo, serão sugeridos comandos para rodar a aplicação:  
			
			\begin{verbatim}
				cd projeto01
				npm run android
				npm run ios
				npm run web
			\end{verbatim}
			
			\begin{figure}
				\centering
				\includegraphics[width=0.7\linewidth]{../images/screenshot001}
				\caption{Saída do comando npx create-expo-app@latest}
				\label{fig:screenshot001}
			\end{figure}
			
			\par Feito isso Abra o diretório que foi criado no vs code: 
			\par \textit{File >  Open Folder > O caminho  do seu diretório}\newline
			
			\par Isso criará um projeto usando as bibliotecas mais atualizadas. No entanto, para garantir a compatibilidade com os códigos feitos em aula, será disponibilizado o arquivo \texttt{package.json}, que lista todas as dependências do projeto.  
			
			\par Para instalar essas dependências, entre no diretório do arquivo citado acima e execute o seguinte comando:  
			
			\begin{verbatim}
				npm install
			\end{verbatim}
			
			\par Pronto! Seu projeto foi criado com sucesso. 
			
		\section{Configurar o ambiente de execução} 
			
			\par O próximo passo é configurar o ambiente de execução onde o sistema rodará. O React Native é capaz de compilar o mesmo projeto para várias plataformas. No entanto, para facilitar o desenvolvimento, utilizaremos a plataforma Expo, que oferece, entre outras funcionalidades, o chamado \textit{hot reload}, permitindo um desenvolvimento mais rápido e sem longas esperas de compilação para diferentes dispositivos.  
			
			\par O guia para realizar essa configuração está disponível em:   \url{https://docs.expo.dev/get-started/set-up-your-environment/} 
			
			\par Primeiro vá na \textit{play store} e instale o \textit{expo go}:
			\begin{figure}
				\centering
				\includegraphics[width=0.4\linewidth]{../images/expogoQrCode}
				\caption{\textit{Expo go} na \textit{play store}}
				\label{fig:expogoqrcode}
			\end{figure}
	
	\chapter{Rodando seu projeto no expo}
		\par No terminal rode o seguinte comando:
			\begin{verbatim}
				npx expo start
			\end{verbatim}
			
	\chapter{Arquivos e diretórios importantes e suas funcionalidades}
		\begin{itemize}
			\item \textbf{tsconfig.json} : Um arquivo de configuração usado em projetos TypeScript para definir as opções do seu compilador.
			\item \textbf{package.json}: Um dos arquivos mais importantes em projetos JavaScript/TypeScript, incluindo projetos React Native. Ele serve como um manifesto do projeto, contendo metadados, dependências, scripts e outras configurações essenciais.
			\item \textbf{\_layout.tsx} : É uma convenção específica usada em projetos que utilizam o Expo Router ou React Navigation para gerenciar a navegação e a estrutura de layouts em aplicações React Native. Ele é especialmente comum em projetos que seguem a estrutura de roteamento baseado em arquivos (file-based routing), onde a organização das rotas é definida pela estrutura de pastas e arquivos.
			\item \textbf{(tabs)}: Qualquer diretório cujo nome esteja entre parênteses é um \textbf{grupo}. \textit{(tabs)} é um grupo especial, porém se preciso for é possível criar novos grupos com outros nomes. Cada grupo precisa de seu próprio arquivo ``\_layout.jsx''
		\end{itemize}
		
	\chapter{Crud em tempo de execução}
	
	\chapter{Crud com persistência}
	
	\chapter{Usando outras fontes}
	
	
\end{document}